% \RequirePackage{snapshot}
% % Version 0.1: 1. Auflage Januar 2018
% %
% %
% %
\AtBeginDocument{\newcommand{\version}{0.1 -- 1.\,Auflage, Januar~2018}}
% % in titelei.tex in den ersten Zeilen % \date{\small{\version}} auskommentieren, wenn \version gesetzt wird sonst \date{}
% %
\documentclass[ 12pt,%
 	 	smallheadings,% Für etwas kleinere Überschriften
		% draft,
		ngerman   		% Neue Rechtschreibung, d.\,h. (Silbentrennung)
		]{scrbook}  	% Eine Klasse für beidseitige Texte mit Kapiteln

\usepackage{xltxtra} 	% This package automatically loads the following packages: fixltx2e, metalogo, xunicode, fontspec
\usepackage{ragged2e}		% Ermöglicht Flattersatz mit Silbentrennung
\usepackage{microtype}	%
\usepackage[babelshorthands]{polyglossia}

\usepackage[automark]{scrlayer-scrpage}

% \usepackage{scrlayer-scrpage}
% \pagestyle{scrheadings}
% \automark[chapter]{chapter}
% \automark*[section]{}

% % Vorgaben der Online-Druckerei
% % Buchrückenstärke 36 mm
\usepackage{geometry}
\geometry{%
	driver={xetex}				%
	,paperheight=216mm		%
	,paperwidth=154mm			%
	,bindingoffset=18mm		%
	% ,tmargin=8mm				%
	% ,textwidth=110mm			% 98mm
	,textwidth=108mm			% 98mm
	,textheight=156mm			% 152mm
	% ,layoutoffset={3mm,3mm}
	% ,rmargin=22mm					%
	,heightrounded				%
  % ,includeheadfoot			%
	% ,headheight=5mm  			%
	% ,headsep=8mm  				%
	% ,foot=18mm  					%
	% ,marginparsep=2mm			%
	% ,marginparwidth=18mm  %
	}


\usepackage{setspace}

% % Typografie Kommandos für das Projekt laden, Anführungszeichen, Euro-Symbol,
\usepackage[autostyle=true,german=guillemets]{csquotes}
\setdefaultlanguage[spelling = new]{german}
\setotherlanguage[spelling = modern]{russian}

\usepackage{verse}
% Auszeichnung des Autors eines Gedichts (s. Kapitel 07 Gedicht »Der Weg von Tschernobyl«)
\newcommand{\attrib}[1]{\nopagebreak{\raggedleft\footnotesize #1\par}}


\usepackage[right]{eurosym}	% Euro-Geldbeträge setzen (mit Euro-Symbol hinter Zahlenwert)

\addto\captionsgerman{%
  \renewcommand{\figurename}{Bild}%
}
\usepackage[labelformat=empty, font=scriptsize, textfont=rm, justification=centering, margin=12pt, aboveskip=6pt]{caption}

\captionsetup[wrapfigure]{labelformat=empty, font=scriptsize, textfont=rm, justification=centering,	margin=12pt, aboveskip=6pt}

% % gefunden in ftp://ftp.dante.de/tex-archive/info/l2tabu/german/l2tabu.pdf 1.8 Nr. 4
\tolerance 1414	% A parameter that tells TeX how much badness is allowable without error. [number] can range from 0 to 10000, and there are no units.
\hbadness 1414	% A parameter that tells TeX at what point to report badness errors (i.e. underfull and overfull error). [number] ranges from 0 to 10000.
\emergencystretch 0.25em	% \emergencystretch (added at TeX3) is used if TeX can not set the paragraph below the \tolerance badness, but rather than make overfull boxes it tries an extra pass "pretending" that every line has an additional \emergencystretch of stretchable glue, this allows the overall badness to be kept below 1000 and stops TeX "giving up" and putting all stretch into one line. So \emergencystretch does not change the setting of "good" paragraphs, it only changes the setting of paragraphs that would have produced over-full boxes.
\hfuzz 0.3pt	% A parameter that allows hbox's to be overfull by [length] before an overfull error occurs.
\widowpenalty=10000	% Penalty for a broken page, with a single line of a paragraph (called "widow") remaining on the top of the succeeding page.
\vfuzz \hfuzz
\raggedbottom	% Command used for top justified within other environments.

\defaultfontfeatures{Mapping=tex-text}
\setmainfont{EB Garamond}
% % Fettschrift definieren: EB Garamond hat keine Fettschrift, die Adobe Garamond Pro keine kyrillischen Glyphen
\setmainfont[BoldFeatures = {Scale=MatchLowercase},
             BoldFont = Adobe Garamond Pro Bold,
             BoldItalicFeatures = {Scale=MatchLowercase},
             BoldItalicFont = Adobe Garamond Pro Bold Italic,
             Ligatures = {TeX,Common}]
            {EB Garamond}
% \setsansfont{Source Sans Pro}
% \setmonofont[Scale=0.9]{Source Sans Pro Light}
\setsansfont{Optima Regular}
\setmonofont[Scale=0.9]{Optima Regular}

\newfontfamily\kolumnenfont[Scale=0.9]{Optima Regular}
\renewcommand*{\headfont}{\kolumnenfont}


\usepackage{pdfpages}
\usepackage{graphicx}
\usepackage{xcolor}
\usepackage{wrapfig}


% % Für die Überarbeitung
% \usepackage[draft, authormarkup=none]{changes} % geänderte Textstellen werden angezeigt
\usepackage[final]{changes} % Text in der neuen Fassung ohne Anzeige der Änderungen

\makeatletter
\@namedef{Changes@AuthorColor}{cyan}
\colorlet{Changes@Color}{cyan}
\makeatother

\usepackage{paralist}	% Extended List Environments für \begin{compactitem} bei \uppertitleback siehe Datei titelei.tex

% % Silbentrennung
\usepackage{hyphenat}
% % % Wortliste Silbentrennung laden
\begin{hyphenrules}{ngerman}
	\hyphenation{}
\end{hyphenrules}
 


\usepackage{tocstyle}
\usetocstyle{noonewithdot}

% \usepackage[firstpage]{draftwatermark}
% \SetWatermarkAngle{45}
% \SetWatermarkColor{blue}
% \SetWatermarkFontSize{30pt}
% \SetWatermarkScale{0.5}
% \SetWatermarkLightness{0}
% \SetWatermarkHorCenter{8cm}
% \SetWatermarkVerCenter
% \SetWatermarkText{\version}

% \usepackage[textsize=tiny]{todonotes}
% \usepackage[obeyDraft]{todonotes}


% %
% TODO PDF Optionen ergänzen/korrigieren
\usepackage{hyperref}
\hypersetup{
		pdfa=true,
    bookmarks=true,bookmarksopen=true,bookmarksopenlevel=1,
	  unicode=true,
	  pdfauthor={Michael Peter Schmidt},
	  pdftitle={2064},
		pdfsubject={1. Auflage 2018},
	  breaklinks,hidelinks,
    colorlinks=false,%
    linkcolor=black,% Print-PDF
    urlcolor=black,%
    % linkcolor=blue,% Online-PDF
    % urlcolor=blue,%
    % citecolor=orange,
    % filecolor=orange,
}

\usepackage{ellipsis}	% korrekte Abstände um \dots Schreibeweise im Quelltext dann \dots\ muss als letztes Paket geladen werden	%Quelle typo-kurz

\KOMAoptions{ %
		cleardoublepage=plain %
}

% % Titelei
\date{\small{\version}}
% \date{}

\newcommand{\buchtitel}{2064}
\newcommand{\buchsubtitel}{Die Geschichte der Cypherpunks}


\title{\buchtitel}
\subtitle{\buchsubtitel}


\dedication{Widmung?}
% \todo{Text Widmung prüfen/überarbeiten}
\author{Michael Peter Schmidt}
\publishers{}

\uppertitleback{ \scriptsize{% 
	\textbf{Bibliografische Informationen der Deutschen Nationalbibliothek:} Die Deutsche Nationalbibliothek verzeichnet diese Publikation in der Deutschen Nationalbibliografie, detaillierte bibliografische Daten sind im Internet über \url{https://portal.dnb.de/} abrufbar.
	
	Umschlaggestaltung, Illustration: 
		
	Lektorat, Korrektorat: 		
		Schriften: EB Garamont von Georg Duffner \url{www.georgduffner.at/ebgaramond/de} und Optima.
		
		}
}

\lowertitleback{\scriptsize{%
	ISBN 978-3-9818594-nn-nn

	\copyright2017 Verlag RMF.Berlin, Rainer-Maria Fritsch, Berlin\\
	1.\,Auflage Januar~2018\\

	Alle Rechte, auch des auszugsweisen Nachdruckes, der auszugsweisen oder vollständigen Wiedergabe, der Speicherung in Datenverarbeitungsanlagen und der Übersetzung, vorbehalten.

	Buchsatz: Rainer-Maria Fritsch, gesetzt mit {\XeLaTeX} und {\KOMAScript}

	Druck: % Pro BUSINESS digital printing Deutschland GmbH, Berlin

	Internet: \url{verlag.rmf.berlin}

	E-Mail: \url{verlag@rmf.berlin}
}
}

% \includeonly{%
% 				% 01_2064_Spiele-Labor,%
% 				% 02_2019_Marianne,%
% 				% 03_2064_osiris-Attacke,%
% 				% 04_2064_Schwachstellen_in_Computern,%
% 		}

\begin{document}
	\frontmatter
		\pagenumbering{Roman}
		\renewcommand{\contentsname}{\textbf{Inhalt}}

		\maketitle
		
		\setkomafont{disposition}{\sffamily\mdseries}
		\tableofcontents
		\setkomafont{disposition}{\sffamily\bfseries}

	\mainmatter
		\pagenumbering{arabic}
		
		\chapter{2064 -- Das \deleted{Spiele-}Labor} % (fold)
\label{cha:2064_das_labor}


Lasse schlenderte froh auf das \deleted{Spiele-}Labor zu, das in der Mitte des Schulgeländes stand\deleted[remark={Das mit dem Bach sollte später beschrieben werden, wenn es um den konkreten Zugang zum Labor geht. Der Leser muss noch nicht gleich wissen, dass es um ein Spiel geht. Labor ist alles: Chemie, Pharmazie, gefährliche Experimente, Tierversuche, ... Dein Leser muss neugierig werden}]{, wo der Bach sich zweigte und auf beiden Seiten um das Labor herum floss}.

Er sah sich um, betrachtete im \replaced{Vorübergehen}{Vorbeigehen} die \deleted[remark={vermeide »völlig«}]{völlig} unterschiedlich gebauten Häuser, aus Lehm, Holz und großen Steinen\added[remark={weitere Materialien?}]{}, mit fantasievoll\added[remark={fantasievoll regt die Fantasie des Lesers recht wenig an. Später greifst du diese Dorfidylle aber nicht mehr auf, oder? Oder in einem späteren Band?}]{} geformten Fenstern und bunten Dächern\added[remark={was soll ich mir unter bunten Dächern vorstellen? Vielfarbige Ziegel, begrünte Dächer? Rote Ziegeldächer, blau glasierte Ziegel?}]{}.\added{Kaum ein Fenster glich dem anderen, rund, eckig, oval, ...}

\replaced{Er mochte es, an ihnen immer wieder ein neues Detail zu entdecken.}{Er mochte es, an ihnen immer wieder ein neues Detail zu entdecken.} 
\replaced{Man}{Und man} konnte den Häusern \deleted{wirklich} ansehen, wer darin arbeitete\added[remark={Eine Gabe die für das Spiel wichtig ist, oder?}]{, wenn man es verstand, ihre Bauweise und die kleinen Details richtig zusammenzufügen}.
Er lächelte.

\added{Bevor er zum Labor ging, besuchte er Alfred.} \replaced{Am Gehege für die Esel wartete er}{Am Eselgehege wartete Alfred} schon und nickte ihm mit seiner langen Schnauze zu.
Lasse zog eine dicke Karotte aus seiner Tasche\added[remark={In was für einer Tasche trägt Lasse eine  dicke(!) Karotte. Aus diesem Satz entsteht kein Bild im Kopf des Lesers}]{} und streckte sie ihm entgegen.
Der schnappte sie vorsichtig aus seiner Hand\added[remark={Der Esel schnappte die Karotte vorsichtig aus seiner Hand [jetzt wird's unklar. Frisst ein Esel eine dicke Karotte in einem Stück? Beißt er ein paar mal ab? Oder verschwindet  er in eine geschützte Ecke, um dort zu fressen? Kurz: wie fressen Esel dicke Karotten?}]{} und drehte sich damit um.

Zur Insel des \deleted{Spiele-}Labors gab es keine Brücke. \replaced{Der Bach, der durch die Siedlung floß, teilte sich hier  und umgab das Labor wie ein schmaler Burggraben. Wer das Labor betreten wollte,}{Um hereinzukommen,} musste man über den Bach springen\added{ können}\replaced{.}{ und }Lasse suchte sich dafür eine Stelle aus, wo er es gerade so \replaced{schaffen konnte}{schaffte}.

Vor dem Haus drückte er seinen Türöffner in der Hosentasche\added[remark={Du führst ein technisches Mittel ein (Türöffner in der Hosentasche), das später nicht mehr vorkommt. Wichtiger scheint mir hier, jetzt dem Leser zu eröffnen, dass es sich um ein Spiele-Labor handelt, dass man z.B. nur betreten kann, wenn die Spieler/innen eine gewisse körperliche Reife haben. Wozu sonst der »Burggraben«?}]{} und die Tür sprang mit einem leisen Klick\added{en} auf.

Es war neun Uhr, eine halbe Stunde vor Beginn des dritten Spieltages.
Er liebte die \replaced[remark={Lass uns ein anderes Wort für TRON finden. Zum einem kommt es später nicht wirklich vor, und ein neuer Name wäre für späteres Marketing/Merchandising auch rechtlich bedeutsam.}]{XXXX}{TRON}-Wochen, ein riesiges Computerspielturnier, bei dem Tausende von Spielern aus vielen Ländern in einer virtuellen Computerwelt mit- und gegeneinander um die Weltherrschaft spielten.

\deleted[remark={Das wird im Buch nach und nach klar werden. Der Leser hat auch noch keine Information, in welcher Zeit Lasse lebt.}]{Und nicht in einer frei erfundenen Welt, sondern in einem ziemlich originalgetreuen Nachbau des Internets der Jahre 2019 bis 2033, eine grandiose Simulation mit unendlich vielen realistischen Details aus dieser Zeit.}

\deleted{Ein Ziel war es, Missionen zu erfüllen, die man bekam: Zum Beispiel }Computer \added{waren} zu erobern \replaced{und}{, sie} unter Kontrolle zu bekommen\added{.} \replaced{Computer}{und zu verteidigen,} in Banken, in Firmen oder auch bei \replaced{Leuten}{irgendjemandem} zu Hause, der \added{irgend}etwas Interessantes machte\added{n}.
\replaced{Wer richtig gut war,}{Man} konnte auch Satelliten, Schiffe oder Flugzeuge \replaced{kapern}{übernehmen}\replaced{ oder }{,} Agenten und Hacker enttarnen.
\deleted[remark=Textumstellung]{Eine sehr beliebte aber auch schwierige Mission war es, bekannte Hacker wie Kevin Mitnick, Adrian Lamo oder Julian Assange zu jagen.}

\replaced{Nicht nur die meisten Punkte bekam, wer geheime Dokumente vor allem von Geheimdiensten, Regierungen oder großen Unternehmen hackte, er konnte diese Informationen teuer an andere Spieler verkaufen.}{Am meisten Punkte gab es, wenn man an geheime Dokumente kam, vor allem wenn sie von Geheimdiensten, Regierungen oder großen Unternehmen kamen.
Die konnte man teuer verkaufen.}

\added[remark={Eine sehr beliebte... ist zu abstrakt. Es wird sonst zu schnell trocken, wie eine Computerspiel-Anleitung (TL;DR) Mache es ein wenig persönlicher.}]{Die Jugendlichen liebten es bekannte Hacker wie Kevin Mitnick, Adrian Lamo oder Julian Assange zu jagen. Es war nur wahnsinnig schwer diese Missionen zu schaffen.}


Lasse hatte unbedingt Julian Assange spielen wollen, aber das wollten wohl zu viele andere auch\added[remark={Das ist keine schlüssige Begründung: das wollten zu viele andere auch. Es ist ihm mangels Fähigkeiten doch eher bisher nicht gelungen, oder?}]{.}
\replaced{Er hatte sich mit der Rolle eines weniger bekannten Hackers abfinden müssen. Immerhin hatte er es geschafft, }{ und so war er ein weniger bekannter Hacker geworden, mit der Mission} Satelliten, fliegende Kampfroboter und andere Waffen in einem Kriegsgebiet zu übernehmen.

Man musste dafür einiges über \added[remark={Hier taucht das schon beschriebene Zeitproblem auf. In welchem Jahr spielt der Anfang der Geschichte?}]{alte} Computersysteme wissen, aber auch die politische Lage von \replaced{vor 50 Jahren}{damals} kennen und wie die Menschen \added{damals} dachten und fühlten.

\replaced[remark={Damals herrschte Krieg im Internet und in vielen Teilen der Welt. Die Welt war voller schrecklicher Konflikte in Regierungen und in Unternehmen, selbst in den Schulen und vielen Familien.}]{Damals herrschte Krieg im Internet und in vielen Teilen der Welt. Die Welt war voller schrecklicher Konflikte in Regierungen und in Unternehmen, selbst in den Schulen und vielen Familien.}{Damals herrschte Krieg im Internet und auch sonst in der Welt.
Auch in Regierungen und in Unternehmen, und oft auch in Schulen und Familien.}
Es war eine völlig andere Welt.

Lasse betrat das Spielzimmer\added[remark={ie soll sich der Leser das Labor vorstellen? Ein Haus mit mehreren Räumen? Gibt es Laborräume für verschiedene Teams? Gibt es nur einen Raum? Bin ich gleich im «Spielzimmer», wenn ich das Labor betrete? Wie hieße er dann? «Spielzimmer» ist zu kindlich.}]{}, in dem Sigur schon gebannt vor seinem Rechner saß und ab und zu etwas tippte.

Lasse: »Hej, moin!«

Er schlug ihm im Vorbeigehen mit der Hand auf die Schulter und ließ sich in seinen Stuhl fallen.


Sigur war sein Flügelmann\added[remark={Flügelmann assoziiert zwei Flügel, linker und rechter Flügel (Fußball), wer ist dann der andere. Findest du eine andere Bezeichnung? Wolltest du ein hierarchisches Verhältnis zwischen den beiden andeuten?}]{} beim Spiel, sein Partner in der aktuellen Mission.
Er war auch sein Freund, wenn auch nicht der beste.

»Er macht zu oft sein eigenes Ding«, dachte Lasse, »hat zu genaue Vorstellungen, wie die Dinge zu sein hätten.«
Das hatte er ihm  schon oft gesagt.

Aber als Flügelmann war Sigur eine tolle Sache\added[remark={Menschen sind keine Sache. War ein toller Kumpel oder suche was besseres}]{}\deleted{, weil er mit seinen 14 Jahren wirklich viel konnte.
Er hatte im Spiel schon jede Menge Computer in Banken und Ölfirmen übernommen, einmal ein ganzes Passagierflugzeug, das aber dann abgestürzt war.
Sogar in Militärcomputern war er schon unterwegs gewesen.}

\added[remark=Textumstellung]{Mit seinen 14 Jahren hatte er im Spiel schon jede Menge Computer in Banken und Ölfirmen übernommen. Selbst in  besonders schwer zu knackenden Militärcomputern war er schon eingedrungen. Einmal war es ihm gelungen, ein Passagierflugzeug zu übernehmen, das dann aber abgestürzt war.}



Lasse: »Was geht, Sig?«

Sigur reagierte nicht und tippte weiter.


Lasse lugte zu ihm herüber: »Ah! Du bist an der Firewall.
Was ist das Problem?«


Nach einiger Zeit sagte\added[remark={Vermeide sagte. Nur in Ausnahmefällen. Zu unlebendig, gerade auf den ersten Seiten. Murmelte, stotterte,  wie wird etwas gesagt, an eine Person gerichtet, vor sich hin,…  erzeuge eine Stimmung,}]{} Sigur ohne vom Bildschirm wegzuschauen: »Ich weiß nicht.
Nur so ein Gefühl.
Irgendetwas stimmt nicht.«



Die Firewall war das Programm auf jedem Computer, das ungebetene Besucher aus dem Internet abhalten sollte.
Eine Art von Filter- oder Wächterprogramm.\added[remark={Hier taucht ein grundsätzliches Problem auf. Die Story wird zum Sachtext. Das ist langweilig. Der ganze Abschnitt über die Firewall gehört in einen Dialog, dass der Leser ahnt, was eine Firewall macht. Du machst das sehr gut im Dialog. Es ist auch für das erste Kapitel unschädlich, noch nicht zu wissen, was eine Firewall ist. Könnte auch ins Glossar, das wir angedacht hatten.}]{}
Jeder Informationsaustausch mit dem Internet geht in beide Richtungen: raus und rein.
Und auf dem Weg rein kann alles Mögliche mit hineinkommen, was Hacker oder Programmierer zu den normalen Daten hinzugefügt haben, was sich dann im Computer festsetzen und für Verwirrung sorgen kann, oder für Schlimmeres.
So etwas soll die Firewall herausfinden und unschädlich machen.
Und wenn sich etwas im Computer festgesetzt hat, auch dafür sorgen, dass es keine Daten wieder herausschicken kann.

Sigur drehte sich zu Lasse um: »Es waren ein paar seltsame Angriffe von einem Yllil-Computer, aber eigentlich nichts Besonders.

Der hat versucht, irgendwo\added[remark={irgendwo? Nein, konkret wo. Eine Firewall die alles blockt ist nicht irgendwo, sondern wo?}]{} hineinzukommen, aber die Firewall hat alles geblockt.
Fühlt sich trotzdem komisch an…«

Lasse: »Yllil? Das klingt Afrikanisch.
– Bei uns stehen heute aber die Chinesen auf dem Plan!« Er grinste.
»Ich habe super Logdateien von einem Angriff auf einen chinesischen Satelliten gefunden, der fast geklappt hätte.
Da finden wir bestimmt was drin.


Logdateien waren Computerdateien, in denen man alles nachlesen konnte, was auf einem Computer so passiert, warum etwas schief gegangen ist, wie etwas geklappt hat und so weiter.\added[remark={Hier wieder zu trockener Sachtext.}]{}

Für Lasse und Sigur waren Logdateien so etwas wie Zeitungen, nur dass darin genau beschrieben stand, was irgendwo\added[remark={Wieder Irgendwo: Wo genau?}]{} passiert \replaced{war}{ist}\deleted[remark={der Vergleich Log-Datei <-> Zeitung ist ok,  muss aber nicht noch vertieft werden}]{ und nicht nur die Meinung eines anderen Menschen darüber}.

Da stand die genaue Zeit, das Programm, was etwas gemacht hatte und was genau passiert war.\added[remark={Das geht sprachlich besser}]{}

\added[remark={Ich habe verstanden, dass dir diese Theaterdialoge wichtig sind. Aber ich finde das eher störend. Ich möchte als Leser im ersten Kapitel in eine Geschichte hineingezogen werden. Die Theaterdialoge für die Zeit um 1913 sind gut, aber hier habe ich meine Zweifel. Lasse und Sigur sind für mein inneres Auge noch nicht lebendig genug, als dass ich mich auf einen Theater-/Drehbuch-Dialog einlassen könnte.
Die Figuren bewegen sich noch nicht richtig. Sie haben noch keine äußere Gestalt. Ich hoffe, du verstehst mich. Auch die Gemeinschaft und zugleich die Spannung zwischen Lasse und Sigur bleibt so farblos}]{} 
Lasse: »\replaced{Auch}{Noch} einen Kakao\added[remark={wo gibt’s im Labor heiße Getränke?}]{} vorher? Heute holen wir das Ding runter!«

Sigur: »Nicht runter! Übernehmen, Daten kopieren, beobachten, unerkannt bleiben, so lange wie möglich.
Das ist unsere Mission.«

Lasse: »Ja, ist gut… Ich weiß \added{doch}.
Die Mission.
Aber schade.
Ich würde so gerne wissen, was TRON macht, wenn wir den Satelliten tatsächlich abstürzen lassen würden.
Das würde mindestens eine politische Krise geben, Vertuschungsversuche, eine Presseschlacht, ein Meer von Lügen, dann Veschwörungstheorien.\added[remark={Sprechen so 14-jährige Schüler?}]{}
Und dann jede Menge neue Missionen, um uns zu schnappen.
\deleted[remark={Beschreiben: grinste über beide Ohren, lächelte unschuldig als könnte er kein Wässerchen trüben,...}]{He he.}«

Sigur schaute ihn streng an: »Ruhig, \added[remark={wie reden sich Jugendliche an?}]{Alter,} ruhig! Wir wollen das Spiel gewinnen, nicht in fünf Minuten rausfliegen.«

% chapter 2064_das_labor (end)

		% == [big-number]#2019# Marianne
\addchap{2019 -- Marianne} % (fold)
\label{cha:2019_marianne}

\textcolor{gray}{Im Jahr 2019 \dots} %\footnote{mittig, grau}

\textsc{Marianne gähnte}. Deutsch-Unterricht, 10.\,Klasse, gedrückte Langeweile im Raum.
Anni neben ihr chattete unter dem Tisch mit ihrem Smartphone.
Sie fischte eine Brotdose aus ihrer Schultasche und platzierte sie vor sich auf den Tisch.
Sie öffnete sie, holte in Papier eingewickelte Butter, Käse und Gurkenscheibchen heraus und ordnete \replaced{alles}{sie} liebevoll nebeneinander an.
Dann nahm sie ein Messer heraus und wollte gerade anfangen eine Brotscheibe mit Butter zu beschmieren, als der Lehrer plötzlich neben ihr stand.


Lehrer: \enquote{Na? Und warum machst du das nicht zu Hause?} Er wippte mit den Füßen.

Marianne schaute ihn mit ruhigem Blick an.
In ihr stieg eine Wut hoch: \enquote{Das ist im Augenblick das Spannendste und Kreativste, was ich tun kann.}

\deleted{Lehrer spitz: }\enquote{Spannender als Faust? Das kann ich mir kaum vorstellen.
Entweder du\dots}

Marianne unterbrach ihn: \enquote{Ich sitze hier seit über einer Stunde rum und muss mir anhören, was Leute in den letzten 200 Jahren über Faust und Mephisto ausgefurzt haben.
Was soll ich damit?
Ich kenne die Leute nicht einmal.
Was hat das mit meinem Leben zu tun?
Hat das überhaupt mit irgendwas heute zu tun?}

Der Lehrer drehte sich abrupt um, atmete ein paar Mal kräftig durch, zeigte in Richtung Tür und schrie \enquote{Raus!}
Er schloss die Augen.
Marianne packte ihre Sachen zusammen, nahm ihre Tasche und ging am Lehrer vorbei aus dem Klassenraum.

\deleted{Er rief ihr hinterher: }\enquote{Du wartest draußen, direkt vor der Tür.}, \added{rief er ihr hinterher}, \enquote{Wir sprechen uns nach dem Unterricht.}
Die Tür krachte ins Schloss.

Marianne machte sich sofort auf den Heimweg.
\enquote{So ein Weichei}, dachte sie, \enquote{wahrscheinlich war das \enquote{ausgefurzt} zu viel für ihn gewesen.} Immerhin hatte er eines der Bücher über Faust, die sie lesen mussten, selbst geschrieben und so konnte er es durchaus persönlich nehmen.
Sie mochte ihn eigentlich, er war ganz cool, als Mensch, aber in der Lehrerrolle\dots Wahrscheinlich mochte er die selbst nicht.
\enquote{Echt ein Weichei}, dachte sie und schüttelte den Kopf.
\enquote{Und was mache ich jetzt mit dem angefangenen Tag?} \enquote{Klar!}, sagte sie laut und schnippte mit den Fingern.

Eine halbe Stunde später saß sie oben auf der großen Treppe rechts neben dem Eingang zum Rathaus Neukölln in einer schattigen Ecke.
Es war ein heißer Tag, um die 28 Grad, und sie trug jetzt ein kurzes, luftiges Kleid, das sie sich auf einem Sprung nach Hause angezogen hatte.
Sie fühlte sich darin ein wenig unwohl, normalerweise trug sie so etwas nicht.
Aber jetzt erfüllte es seinen Zweck.

Sie zog einen noch eingeschweißten Laptop aus ihrem Stoffbeutel.
Er war unbenutzt, aber nicht mehr auf dem neuesten Stand der Technik.
Er war aus dem Jahr 2009 und hatte noch nicht die Überwachungschips eingebaut, die inzwischen in allen neuen Computern zu finden waren.
\deleted{Durch diese Chips konnten Geheimdienste Computer über das Internet unbemerkt fernsteuern, Daten mitlesen oder über die Kamera und das Mikrofon die Umgebung überwachen. Einige Computer konnte man darüber sogar über das Netz anschalten.}
Durch diese Chips konnten Geheimdienste über das Internet unbemerkt Daten mitlesen, den Computer fernsteuern und über die Kamera und das Mikrofon die Umgebung überwachen. Einige Computer konnten sogar über das Netz heimlich eingeschaltet werden.


Oskar, einer ihrer Freunde, hatte in der Firma, in der er arbeitete, zehn dieser Computer entdeckt.
Niemand dort schien davon zu wissen, sie waren in \replaced[remark={Plural scheint mir flüssiger}]{den Lagerlisten}{der Lagerliste} nicht aufgeführt, und so hatte er sie einfach mitgenommen.
Er war mit einem Gabelstapler ins Lager gefahren und hatte eine Menge alter Kartons zusammen mit den Computern aufgeladen und war einfach damit herausgefahren.
Dem Lagerleiter hatte er gesagt, dass er die Kartons für ein Schülerprojekt brauchen würde.
Was für ein Geschenk in Zeiten, in denen die Seriennummer jedes Computers von der Produktion bis zur Müllhalde verfolgt und zusammen mit den E-Mail-Adressen, Telefonnummern, Aufenthaltsorten und Fotos der jeweiligen Besitzer abgespeichert wurde.

Sie steckte einen USB-Stick mit der Aufschrift \enquote{Tails 4.5} in ihr Notebook und drückte den Anschaltknopf.
Tails war das eines der sichersten Betriebssysteme.
Es hinterließ keine bleibenden Spuren auf dem Rechner, auf dem es \replaced{gestartet worden war}{lief}.
Keine Hinweise auf den Benutzer, keine Informationen über seinen Aufenthaltsort.
Wenn man den Stick wieder abzog, blieben nichts von dem übrig, was auf dem Computer gemacht worden war.
Manche Menschen vertrauten sogar mit ihren Leben darauf, dass das sicher funktionierte.

Ihre Knie zitterten leicht.
Tails sollte ihr dabei helfen, das zu tun, was sie jetzt vorhatte.
Sie hatte sich auf dem Heimweg von der Schule entschlossen, nicht noch einen weiteren Tag zu warten.
Sie gab ein langes Passwort ein\footnote{Wo wird in Tails ein langes Passwort eingegeben?} und öffnete \deleted{dann} mit einem Doppelklick das Terminal-Programm von Tails.
Manche nannten es Kommando-Zeile: Man gab ein Kommando ein und der Computer antwortete mit Text.
Das war alles.
Aber die Kommandozeile hatte es in sich.
Alle Hacker, die sie kannte, arbeiteten fast ausschließlich damit.
Es gab mächtige Befehle für alles, was man machen wollte.
Und für das was sie vorhatte, waren Maus und Fenster viel zu langsam.
Im Terminal ging alles viel schneller und \replaced{präziser}{genauer} als mit allen anderen Programmen.

Und das war jetzt wichtig: schnell sein, genau sein.
% chapter 2019_marianne (end)

		
		%== [big-number]#2064# TRON-Attacke
\addchap{2064 -- \textsc{Osiris}-Attacke} % (fold)
\label{cha:2064_osiris_attacke}



\textsc{Sigur erhob sich} und drehte sich zu Lasse.

Sigur: \enqoute{Es ist Zeit.}

Beide gingen zur Raummitte, wo vier große Computer-Bildschirmscheiben über einem Tisch schwebten.
Auf dem Tisch lagen zwei Tastaturen, vier fast faustgroße Controller und zwei große Glaskugeln, die nach unten hin offen waren.
Lasse und Sigur setzten sich auf ihre Sessel und nahmen sich die Tastaturen.

\enquote{9:30 Uhr! Es geht los.}, rief Lasse.
Beide saßen gespannt vor ihren Monitoren und schauten auf die Aktivitätsanzeige.
Sigur nickte entschlossen mit dem Kopf.

Lasse: \enquote{Okay.
Alles ruhig.
Dann wollen wir mal loslegen.} 

Er stieß einen Schrei aus.

Lasse: \enquote{S H I T! Sig, mach die Firewall zu, wir bekommen gerade einen massiven Angriff, quantum parallel.}

Sigur hackte mit seiner rechten Hand schnell vier, fünfmal auf seine Tastatur und fixierte  mit seinen Augen den Bildschirm.

Sigur: \enquote{Done! Ist zu! Wer war das?} 

Lasse zuckt mit den Schultern.
\enquote{Keine Ahnung.}

Sigur schrie: \enquote{WAAAASSSSS???!!!}

Lasse: \enquote{Was??}

Sigur trommelte auf seine Tastatur: \enquote{Shit! \dots\  Shit!}

Lasse: \enquote{Sag schon.} 

Er sprang auf und stellte sich neben Sigur.

Lasse: \enquote{Was ist los?}

Sigur hektisch: \enquote{Die Firewall ist wieder offen!
Einfach so \dots\
Und ich bin \dots\ draußen, weg, kann nichts tun.
Nicht mal tippen.
Nichts.}
Er schlug mit der Faust auf seine Tastatur.

Lasse: \enquote{Shit!}

Sigur: \enquote{Fuck!}

Lasse: \enquote{Zieh den Stecker!}

Sigur: \enquote{Das dürfen wir nicht.
Gegen die Regeln.}

Lasse: \enquote{Scheiß drauf.
\dots\  Okay.
Gut, gut.
Dann eben nicht.} 

Sigur schmiss die Tastatur von sich, stand auf und schlug ein paar Mal mit der Faust gegen die Wand.

\enquote{Was war das?}, fragte Lasse.

\enquote{WER war das?}, fragte Sigur immer noch mit dem Gesicht zur Wand.

Mit einem leisen Klacken öffnete sich die Tür.
Beide drehten sich gleichzeitig dorthin.
Lilly lugte mit einem breiten Grinsen herein: \enquote{Hi, hi! Zu langsam, Jungs\dots}

Lasse: \enquote{Oh nein! Nicht.
Bitte nicht \dots\  Nicht schon wieder \dots} 

Lilly war Lasses kleine Schwester und vor zwei Monaten 12 Jahre alt geworden.
Sie spielte \textsc{Osiris} erst seit einem halben Jahr, aber sie hatte wirklich Talent dafür.
Schon bei ihrem ersten Spiel war es ihr gelungen, die Dokumente zu finden, die Edward Snowden im Jahr 2013 von der NSA, dem größten amerikanischen Geheimdienst mitgenommen hatte.
Die Snowden-Dokumente hatten damals weltweit für großes Aufsehen gesorgt.

\textsc{Osiris} reagierte damals mit einige Missionen, um Lilly zu fangen, und viele Spieler nahmen dann die Jagd auf Lilly auf.
Aber sie schaffte es zu überleben, und niemandem war es gelungen, ihr die Snowden-Dokumente wieder abzujagen.

% chapter 2064_osiris_attacke (end)
		
		\include{04_2064_Schwachstellen_in_Computern}
	

	\backmatter
		\begin{appendix}
			\chapter*{Weitere Informationen und Kontakt} % (fold)
\label{cha:mehr_informationen}

% \addcontentsline{toc}{chapter}{Weitere Informationen und Kontakt}


% \noindent Weitere Informationen zum Buch finden Sie auf
%
% \noindent \url{https://verlag.rmf.berlin}\\[12pt]


\noindent Weitere Informationen zum Autor Michael Peter Schmidt finden Sie auf unserer Website \url{https://verlag.rmf.berlin}\\[12pt]

% TODO Informationen zur Grafikerin einfügen

\noindent Für Ihre Anregungen, Fragen und Kritik schreiben Sie bitte eine E-Mail an \url{verlag@rmf.berlin}

\vspace{60pt}

\noindent Dieses Buch wurde gesetzt mit {\XeLaTeX} und {\KOMAScript} -- Open Source Textsatz. Nähere Informationen zu {\LaTeX} und  {\XeLaTeX} unter \\\url{www.dante.de} und zu {\KOMAScript} unter \url{www.komascript.de}.

% Für technische Fragen senden Sie bitte eine E-Mail an \url{texlatex@rmf.berlin}



% chapter mehr_informationen (end)
		\end{appendix}
	% \listofchanges

\end{document}
