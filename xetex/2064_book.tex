% \RequirePackage{snapshot}
% % Version 0.1: 1. Auflage Januar 2018
% %
% %
% %
\AtBeginDocument{\newcommand{\version}{0.1 -- 1.\,Auflage, Januar~2018}}
% % in titelei.tex in den ersten Zeilen % \date{\small{\version}} auskommentieren, wenn \version gesetzt wird sonst \date{}
% %
\documentclass[ 12pt,%
 	 	smallheadings,% Für etwas kleinere Überschriften
		% draft,
		ngerman   		% Neue Rechtschreibung, d.\,h. (Silbentrennung)
		]{scrbook}  	% Eine Klasse für beidseitige Texte mit Kapiteln

\usepackage{xltxtra} 	% This package automatically loads the following packages: fixltx2e, metalogo, xunicode, fontspec
\usepackage{ragged2e}		% Ermöglicht Flattersatz mit Silbentrennung
\usepackage{microtype}	%
\usepackage[babelshorthands]{polyglossia}

\usepackage[automark]{scrlayer-scrpage}

% \usepackage{scrlayer-scrpage}
% \pagestyle{scrheadings}
% \automark[chapter]{chapter}
% \automark*[section]{}

% % Vorgaben der Online-Druckerei
% % Buchrückenstärke 36 mm
\usepackage{geometry}
\geometry{%
	driver={xetex}				%
	,paperheight=216mm		%
	,paperwidth=154mm			%
	,bindingoffset=18mm		%
	% ,tmargin=8mm				%
	% ,textwidth=110mm			% 98mm
	,textwidth=108mm			% 98mm
	,textheight=156mm			% 152mm
	% ,layoutoffset={3mm,3mm}
	% ,rmargin=22mm					%
	,heightrounded				%
  % ,includeheadfoot			%
	% ,headheight=5mm  			%
	% ,headsep=8mm  				%
	% ,foot=18mm  					%
	% ,marginparsep=2mm			%
	% ,marginparwidth=18mm  %
	}


\usepackage{setspace}

% % Typografie Kommandos für das Projekt laden, Anführungszeichen, Euro-Symbol,
\usepackage[autostyle=true,german=guillemets]{csquotes}
\setdefaultlanguage[spelling = new]{german}
\setotherlanguage[spelling = modern]{russian}

\usepackage{verse}
% Auszeichnung des Autors eines Gedichts (s. Kapitel 07 Gedicht »Der Weg von Tschernobyl«)
\newcommand{\attrib}[1]{\nopagebreak{\raggedleft\footnotesize #1\par}}


\usepackage[right]{eurosym}	% Euro-Geldbeträge setzen (mit Euro-Symbol hinter Zahlenwert)

\addto\captionsgerman{%
  \renewcommand{\figurename}{Bild}%
}
\usepackage[labelformat=empty, font=scriptsize, textfont=rm, justification=centering, margin=12pt, aboveskip=6pt]{caption}

\captionsetup[wrapfigure]{labelformat=empty, font=scriptsize, textfont=rm, justification=centering,	margin=12pt, aboveskip=6pt}

% % gefunden in ftp://ftp.dante.de/tex-archive/info/l2tabu/german/l2tabu.pdf 1.8 Nr. 4
\tolerance 1414	% A parameter that tells TeX how much badness is allowable without error. [number] can range from 0 to 10000, and there are no units.
\hbadness 1414	% A parameter that tells TeX at what point to report badness errors (i.e. underfull and overfull error). [number] ranges from 0 to 10000.
\emergencystretch 0.25em	% \emergencystretch (added at TeX3) is used if TeX can not set the paragraph below the \tolerance badness, but rather than make overfull boxes it tries an extra pass "pretending" that every line has an additional \emergencystretch of stretchable glue, this allows the overall badness to be kept below 1000 and stops TeX "giving up" and putting all stretch into one line. So \emergencystretch does not change the setting of "good" paragraphs, it only changes the setting of paragraphs that would have produced over-full boxes.
\hfuzz 0.3pt	% A parameter that allows hbox's to be overfull by [length] before an overfull error occurs.
\widowpenalty=10000	% Penalty for a broken page, with a single line of a paragraph (called "widow") remaining on the top of the succeeding page.
\vfuzz \hfuzz
\raggedbottom	% Command used for top justified within other environments.

\defaultfontfeatures{Mapping=tex-text}
\setmainfont{EB Garamond}
% % Fettschrift definieren: EB Garamond hat keine Fettschrift, die Adobe Garamond Pro keine kyrillischen Glyphen
\setmainfont[BoldFeatures = {Scale=MatchLowercase},
             BoldFont = Adobe Garamond Pro Bold,
             BoldItalicFeatures = {Scale=MatchLowercase},
             BoldItalicFont = Adobe Garamond Pro Bold Italic,
             Ligatures = {TeX,Common}]
            {EB Garamond}
% \setsansfont{Source Sans Pro}
% \setmonofont[Scale=0.9]{Source Sans Pro Light}
\setsansfont{Optima Regular}
\setmonofont[Scale=0.9]{Optima Regular}

\newfontfamily\kolumnenfont[Scale=0.9]{Optima Regular}
\renewcommand*{\headfont}{\kolumnenfont}


\usepackage{pdfpages}
\usepackage{graphicx}
\usepackage{xcolor}
\usepackage{wrapfig}


% % Für die Überarbeitung
% \usepackage[draft, authormarkup=none]{changes} % geänderte Textstellen werden angezeigt
\usepackage[final]{changes} % Text in der neuen Fassung ohne Anzeige der Änderungen

\makeatletter
\@namedef{Changes@AuthorColor}{cyan}
\colorlet{Changes@Color}{cyan}
\makeatother

\usepackage{paralist}	% Extended List Environments für \begin{compactitem} bei \uppertitleback siehe Datei titelei.tex

% % Silbentrennung
\usepackage{hyphenat}
% % % Wortliste Silbentrennung laden
\begin{hyphenrules}{ngerman}
	\hyphenation{}
\end{hyphenrules}
 


\usepackage{tocstyle}
\usetocstyle{noonewithdot}

% \usepackage[firstpage]{draftwatermark}
% \SetWatermarkAngle{45}
% \SetWatermarkColor{blue}
% \SetWatermarkFontSize{30pt}
% \SetWatermarkScale{0.5}
% \SetWatermarkLightness{0}
% \SetWatermarkHorCenter{8cm}
% \SetWatermarkVerCenter
% \SetWatermarkText{\version}

% \usepackage[textsize=tiny]{todonotes}
% \usepackage[obeyDraft]{todonotes}


% %
% TODO PDF Optionen ergänzen/korrigieren
\usepackage{hyperref}
\hypersetup{
		pdfa=true,
    bookmarks=true,bookmarksopen=true,bookmarksopenlevel=1,
	  unicode=true,
	  pdfauthor={Michael Peter Schmidt},
	  pdftitle={2064},
		pdfsubject={1. Auflage 2018},
	  breaklinks,hidelinks,
    colorlinks=false,%
    linkcolor=black,% Print-PDF
    urlcolor=black,%
    % linkcolor=blue,% Online-PDF
    % urlcolor=blue,%
    % citecolor=orange,
    % filecolor=orange,
}

\usepackage{ellipsis}	% korrekte Abstände um \dots Schreibeweise im Quelltext dann \dots\ muss als letztes Paket geladen werden	%Quelle typo-kurz

\KOMAoptions{ %
		cleardoublepage=plain %
}

% % Titelei
\date{\small{\version}}
% \date{}

\newcommand{\buchtitel}{2064}
\newcommand{\buchsubtitel}{Die Geschichte der Cypherpunks}


\title{\buchtitel}
\subtitle{\buchsubtitel}


\dedication{Widmung?}
% \todo{Text Widmung prüfen/überarbeiten}
\author{Michael Peter Schmidt}
\publishers{}

\uppertitleback{ \scriptsize{% 
	\textbf{Bibliografische Informationen der Deutschen Nationalbibliothek:} Die Deutsche Nationalbibliothek verzeichnet diese Publikation in der Deutschen Nationalbibliografie, detaillierte bibliografische Daten sind im Internet über \url{https://portal.dnb.de/} abrufbar.
	
	Umschlaggestaltung, Illustration: 
		
	Lektorat, Korrektorat: 		
		Schriften: EB Garamont von Georg Duffner \url{www.georgduffner.at/ebgaramond/de} und Optima.
		
		}
}

\lowertitleback{\scriptsize{%
	ISBN 978-3-9818594-nn-nn

	\copyright2017 Verlag RMF.Berlin, Rainer-Maria Fritsch, Berlin\\
	1.\,Auflage Januar~2018\\

	Alle Rechte, auch des auszugsweisen Nachdruckes, der auszugsweisen oder vollständigen Wiedergabe, der Speicherung in Datenverarbeitungsanlagen und der Übersetzung, vorbehalten.

	Buchsatz: Rainer-Maria Fritsch, gesetzt mit {\XeLaTeX} und {\KOMAScript}

	Druck: % Pro BUSINESS digital printing Deutschland GmbH, Berlin

	Internet: \url{verlag.rmf.berlin}

	E-Mail: \url{verlag@rmf.berlin}
}
}

% \includeonly{%
% 				% 01_2064_Spiele-Labor,%
% 				% 02_2019_Marianne,%
% 				% 03_2064_osiris-Attacke,%
% 				% 04_2064_Schwachstellen_in_Computern,%
% 		}

\begin{document}
	\frontmatter
		\pagenumbering{Roman}
		\renewcommand{\contentsname}{\textbf{Inhalt}}

		\maketitle
		
		\setkomafont{disposition}{\sffamily\mdseries}
		\tableofcontents
		\setkomafont{disposition}{\sffamily\bfseries}

	\mainmatter
		\pagenumbering{arabic}
		
		\addchap{2064 -- Das Labor} % (fold)
\label{cha:2064_das_labor}

\textcolor{gray}{Im Jahr 2064 \dots}% . [remark={Das sollte LaTeX irgendwie ausgegraut in die Mitte schreiben. Es ist wichtig für die Geschichte hier das Jahr zu haben und ich will es nicht im Text haben, ich dachte der Titel reicht, dort ist das Jahr groß gedruckt.}]

\textsc{Lasse schlenderte froh} auf das Labor zu, das in der Mitte des Schulgeländes stand. 
\replaced{Ein Bach schlängelte sich zwischen großen vereinzelt stehenden Bäumen und den aus Lehm, Holz, großen Steinen, Stroh und Glas gebauten Häusern, Kaum ein Haus glich dem anderen. }{Sein Blick strich über die sanften Hügel, die Häuser, die vielen allein stehenden Bäume und den Bach, der sich dazwischen schlängelte.}

\deleted{Im Vorübergehen betrachtete er die einzelnen Gebäude, die aus Lehm, Holz, großen Steinen, Stroh und Glas gebaut waren und ganz unterschiedlich aussahen. Kaum ein Fenster glich dem anderen.} Die Ziegel der Dächer spielten \replaced{von}{in verschiedenen Farben: meist} zinnoberrot\replaced{ über}{,} blau glasiert und dunkelgrün. \added{Die Fenster waren Ausdruck einer Freude an geometrischen Formen: Dreiecke, Quadrate, Polygone, Kreise, Ovale.} Alles war bunt, und doch formte es sich zu einem harmonischen Ganzen.
\replaced{Immer wieder}{Überall} entdeckte er \deleted{immer wieder} neue Details.
Er lächelte.

\replaced{Oft, b}{B}evor er zum Labor ging, besuchte er \deleted{oft} Alfred.
Der wartete heute schon am Zaun des Eselgeheges und nickte ihm mit seiner langen Schnauze zu.
Lasse zog eine dicke Karotte aus seiner Hosentasche und streckte sie ihm entgegen.
Der schnappte sie vorsichtig aus seiner Hand,\deleted{ nahm sie ganz in sein Maul,} \deleted{und} drehte sich kauend \deleted{damit} um \added{ und trabte in eine geschützte Ecke}.
Lasse schüttelte lächelnd den Kopf.

Das Labor stand zwischen zwei großen Bäumen auf einer Insel, \replaced{die vom Bach umschlossen war.}{um die der Bach an beiden Seiten herumfloss.}
Es war ein großer runder Bau mit einer flachen Kuppel und \added{nach außen} geschwungenen Wänden, die auf viele runde Zimmer \added{in seinem Inneren} hindeuteten.
Es gab keine Brücke.
Um auf die Insel zu kommen, musste Lasse über den Bach springen.
Er suchte sich dafür eine Stelle aus, wo er es gerade so schaffte.

Das Labor war der Ort, an dem eine Landschaft mit Bergen, Städten, Wäldern, Straßen,  Flüssen und einem Stausee geschaffen werden konnte. Welche Wirkung würde ein Erdbeben auf die Staumauer haben?\footnote{Meine kläglichen Versuche, von \enquote{man} wegzukommen}
\deleted{Das Labor war der Ort, an dem man experimentieren konnte: Man konnte ein Landschaft mit Bergen, Städten, Wäldern, Straßen und Flüssen erschaffen, einen Stausee bauen und den dann durch ein Erdbeben brechen lassen.}
Neue Materialien wurden erfunden und in ihren Fähigkeiten und in ihren Reaktionen auf andere Stoffe getestet.
\deleted{Man konnte neue Materialien erfinden und ausprobieren, was man mit ihnen machen konnte, wie sie mit anderen Materialien reagierten.}
Expeditionen ins All wurden ausgerüstet und ferne Galaxien erforscht.\deleted{Oder eine Expedition zu fernen Planeten machen und diese erforschen.}
Wer bereit war, in die Geschichte zurückzureisen, konnte dort in einen Menschen schlüpfen und hautnah frühere Zeiten erleben.\deleted{Man konnte auch in der Geschichte zurückreisen, in alle möglichen Zeiten, und dort verschiedene Rollen annehmen, die Eigenheiten der Zeit kennenlernen und Dinge ändern.}
\deleted{Natürlich ging das alles nur in Computer-Simulationen.
Aber die waren so realistisch,} Diese Computer--Simulationen waren so realistisch, dass einige  nur  älteren Schülerinnen zugänglich waren.

Lasse lief auf den Eingang des Labors zu. Das Hologramm einer Fee\footnote{ich bin gegen die Fee. Das Labor ist ein Ort besonderer Herausforderungen. Wer oder was könnte ein passender Türwächter sein? Ein abstraktes Symbol wie: Erlenmeyerkolben oder eine der Figuren der Akademie?} erschien, die ein wenig größer war als er selbst.

\deleted{Fee: }»Dein Code, Lasse!«

Lasse berührte ein Amulett auf seiner Brust und murmelte etwas.

\deleted{Fee: »Danke.«}

\replaced{»Danke«, quittierte d}{D}ie Fee \added{und }verschwand\replaced{. D}{ und d}ie Türe öffnete sich mit einem leisen Klicken.

Im Gebäude kreuzte er den weiten Innenhof und lief auf einen runden Eingang zu, über dem \replaced{ihre Namen}{die Worte} »Lasse und Sigur« leuchteten.
Es war neun Uhr, eine halbe Stunde vor Beginn des dritten Spieltages.
Er liebte die \textsc{Osiris}-Wochen, ein riesiges Computerspiel-Turnier, bei dem Tausende von Spielern aus vielen Ländern in einer virtuellen \replaced{W}{Computerw}elt mit- und gegeneinander um die Weltherrschaft spielten.
Und nicht in einer frei erfundenen Welt, sondern in einem ziemlich originalgetreuen Nachbau des Internets der Jahre 2019 bis 2033.

Dort waren Computer zu erobern und zu verteidigen\replaced{. Computer}{,} in Banken, großen Firmen oder auch bei Leuten zu Hause, die irgendetwas Interessantes machten.
Wer richtig gut war, konnte auch Satelliten, Schiffe oder Flugzeuge übernehmen oder Agenten und Hacker enttarnen.
Die Jugendlichen liebten es, bekannte Hacker wie Kevin Mitnick, Adrian Lamo oder Julian Assange zu jagen.
Es war nur wahnsinnig schwer diese Missionen zu schaffen.
Sehr begehrt waren geheime Dokumente, vor allem von Geheimdiensten, Regierungen oder großen Unternehmen. \replaced{Solche wichtigen Dokumente}{Die} konnte man teuer an andere Spieler verkaufen.

Lasse hatte eigentlich Julian Assange spielen wollen, aber \replaced{das war ihm noch nicht gestattet worden}{das hatte nicht geklappt}.\footnote{Hier ein Wink zur späteren Akademie. Ist auch schlüssiger zum folgenden \enquote{abfinden}}
Er musste sich mit der Rolle eines weniger bekannten Hackers abfinden.
Immerhin zählten zu seinen Aufgaben, Satelliten, fliegende Kampfroboter und andere gefährliche Waffen in einem Kriegsgebiet zu übernehmen und damit Missionen zu erfüllen.

Man musste dafür einiges über Computersysteme wissen, aber auch die politische Lage von vor 50 Jahren kennen und wie die Menschen dachten und fühlten.

Damals herrschte Krieg im Internet und in vielen Teilen der Welt.
Überall gab es Gewalt und Konflikte, in Regierungen, in Unternehmen, sogar an Schulen und in Familien.
Es war eine völlig andere Welt.
Sie war Lasse fremd, aber gerade darum auch \added{so} interessant.

Lasse betrat seinen Spielraum, in dem Sigur schon gebannt vor seinem Rechner saß und ab und zu etwas tippte.

Lasse: »Hej, Alter!«

Er schlug ihm im Vorbeigehen mit der Hand auf die Schulter und ließ sich in seinen Stuhl fallen.

Sigur war sein Flügelmann im Spiel, sein Partner in der aktuellen Mission. %\footnote{WikiPedia: In Luftstreitkräften wird das Wort auch heute noch als taktischer Begriff benutzt, jedoch meist in der englischen Version Wingman. Ein Wingman ist ein Pilot, der einen anderen Piloten in einem feindlichen Umfeld unterstützt.}
Er war auch sein Freund, wenn auch nicht der beste.
»Er macht zu oft sein eigenes Ding«, dachte Lasse, »hat zu genaue Vorstellungen, wie die Dinge zu sein hätten.«
Das hatte er ihm  schon oft gesagt.
%\footnote{Sigur ist hier in seinem Charakter sehr genau und penibel, das führt dann später dazu, dass er über die Grenzen hinaus geht. Lasse probiert die Grenzen aus und kennt sich deswegen damit besser aus. \enquote{Er macht zu oft sein eigenes Ding}ist nicht sauber ausgedrückt, Lasse meint, er drückt zu oft das durch was er will, aber ich will diesen Fehler gerne drin lassen, weil Lasse auch erst 14 ist.}

Aber als Flügelmann war Sigur das Beste, was ihm passieren konnte\replaced{. M}{, weil er m}it seinen 14 Jahren \replaced{war er wirklich schon sehr weit gekommen.}{wirklich viel konnte}.
Er hatte im Spiel schon jede Menge Computer in Banken und Ölfirmen übernommen. Selbst in besonders schwer zu knackende Militärcomputer war er \deleted{schon} eingedrungen.
Einmal war es ihm gelungen, ein Passagierflugzeug zu übernehmen, das dann aber abgestürzt war.

Lasse: »Sig, was geht?«

Sigur reagierte nicht und tippte weiter.

Lasse lugte zu ihm herüber: »Ah! Du bist an der Firewall.
Was ist das Problem?«

Nach einiger Zeit murmelte Sigur ohne vom Bildschirm wegzuschauen: »Ich weiß nicht.
Nur so ein Gefühl.
Irgendetwas stimmt nicht.«

Lasse: »Ist jemand in unsere Computer eingedrungen?
Haben wir ein Loch in der Firewall?
Ist etwas gestohlen worden?«

Sigur schüttelte den Kopf und drehte sich zu Lasse um: »Es waren ein paar seltsame Angriffe von einem Yllil-Computer, aber eigentlich nichts Besonders.
Der hat versucht, an unsere Passwort-Dateien zu kommen, aber die Firewall hat alles geblockt.
Fühlt sich trotzdem komisch an\dots«

Lasse: »Yllil? Das klingt Afrikanisch.
-- Bei uns stehen heute aber die Chinesen auf dem Plan!« Er grinste.
»Ich habe super Logdateien von einem Angriff auf einen chinesischen Satelliten gefunden, der fast geklappt hätte.
Da drin ist alles haarklein verzeichnet, was der Hacker, äh, die Hackerin, gemacht hat, um \replaced{r}{hin}einzukommen.
Echt schlau.
Eine echte Häckse.« 

Er lächelte.

Lasse: »Und toll zu lesen.
Da finden wir bestimmt etwas für uns drin.«

%[remark={Warum der Kommentar hier? Der erste Theater-Dialog kommt im 8. Kapitel: Ich habe verstanden, dass dir diese Theaterdialoge wichtig sind. Aber ich finde das eher störend. Ich möchte als Leser im ersten Kapitel in eine Geschichte hineingezogen werden. Die Theaterdialoge für die Zeit um 1913 sind gut, aber hier habe ich meine Zweifel. Lasse und Sigur sind für mein inneres Auge noch nicht lebendig genug, als dass ich mich auf einen Theater-/Drehbuch-Dialog einlassen könnte.
%Die Figuren bewegen sich noch nicht richtig. Sie haben noch keine äußere Gestalt. Ich hoffe, du verstehst mich. Auch die Gemeinschaft und zugleich die Spannung zwischen Lasse und Sigur bleibt so farblos}]

Lasse: »Noch einen Kakao vorher?« Er zeigte auf den kleinen Getränkeautomaten, der an der Wand hing. »Heute holen wir das Ding runter!«

Sigur: »Nicht runter! Übernehmen, Daten kopieren, beobachten, unerkannt bleiben, so lange wie möglich.
Das ist unsere Mission.«

Lasse: »Ja, ist gut… Ich weiß doch.
Die Mission.
Aber schade.
Ich würde so gerne wissen, was \textsc{Osiris} macht, wenn wir den Satelliten tatsächlich abstürzen lassen würden.
Dann wären wir überall in den Nachrichten und weit oben auf der Fahndungsliste.
Das würde jede Menge neue Missionen geben, um uns zu schnappen.«

Er grinste über beide Ohren: »He, he.«

Sigur schaute ihn streng an: »Alter, ruhig! Wir wollen das Spiel gewinnen, nicht in fünf Minuten rausfliegen.«

% chapter 2064_das_labor (end)

		% == [big-number]#2019# Marianne
\addchap{2019 -- Marianne} % (fold)
\label{cha:2019_marianne}

\textcolor{gray}{Im Jahr 2019 \dots} %\footnote{mittig, grau}

\textsc{Marianne gähnte}. Deutsch-Unterricht, 10.\,Klasse, gedrückte Langeweile im Raum.
Anni neben ihr chattete unter dem Tisch mit ihrem Smartphone.
Sie fischte eine Brotdose aus ihrer Schultasche und platzierte sie vor sich auf den Tisch.
Sie öffnete sie, holte in Papier eingewickelte Butter, Käse und Gurkenscheibchen heraus und ordnete alles liebevoll nebeneinander an.
Dann nahm sie ein Messer heraus und wollte gerade anfangen eine Brotscheibe mit Butter zu beschmieren, als der Lehrer plötzlich neben ihr stand.

Lehrer: \enquote{Na? Und warum machst du das nicht zu Hause?} Er wippte mit den Füßen.

Marianne schaute ihn mit ruhigem Blick an.
In ihr stieg eine Wut hoch: \enquote{Das ist im Augenblick das Spannendste und Kreativste, was ich tun kann.}

\deleted{Lehrer spitz: }\enquote{Spannender als Faust? Das kann ich mir kaum vorstellen.
Entweder du\dots}

Marianne unterbrach ihn: \enquote{Ich sitze hier seit über einer Stunde rum und muss mir anhören, was Leute in den letzten 200 Jahren über Faust und Mephisto ausgefurzt haben.
Was soll ich damit?
Ich kenne die Leute nicht einmal.
Was hat das mit meinem Leben zu tun?
Hat das überhaupt mit irgendwas heute zu tun?}

Der Lehrer drehte sich abrupt um, atmete ein paar Mal kräftig durch, zeigte in Richtung Tür und schrie \enquote{Raus!}
Er schloss die Augen.
Marianne packte ihre Sachen zusammen, nahm ihre Tasche und ging am Lehrer vorbei aus dem Klassenraum.

Lehrer: \enquote{Du wartest draußen, direkt vor der Tür.}, rief er ihr hinterher, \enquote{wir sprechen uns nach dem Unterricht.}
Die Tür krachte ins Schloss.

Marianne machte sich sofort auf den Heimweg.
\enquote{So ein Weichei}, dachte sie, \enquote{wahrscheinlich war das \enquote{ausgefurzt} zu viel für ihn gewesen.}
Immerhin hatte er eines der Bücher über Faust, die sie lesen mussten, selbst geschrieben und so konnte er es durchaus persönlich nehmen.
Sie mochte ihn eigentlich, er war ganz cool, als Mensch, aber in der Lehrerrolle\dots Wahrscheinlich mochte er die selbst nicht.
\enquote{Echt ein Weichei}, dachte sie wieder und schüttelte den Kopf.
\enquote{Und was mache ich jetzt mit dem angefangenen Tag?} \enquote{Klar!}, sagte sie laut und schnippte mit den Fingern.

Eine halbe Stunde später saß sie oben auf der großen Treppe rechts neben dem Eingang zum Rathaus Neukölln in einer schattigen Ecke.
Es war ein heißer Tag, um die 28 Grad, und sie trug jetzt ein kurzes, luftiges Kleid, das sie sich auf einem Sprung nach Hause angezogen hatte.
Sie fühlte sich darin ein wenig unwohl, normalerweise trug sie so etwas nicht.
Aber jetzt erfüllte es seinen Zweck.

Sie zog einen noch eingeschweißten Laptop aus ihrem Stoffbeutel.
Er war unbenutzt, aber nicht mehr auf dem neuesten Stand der Technik.
Er war aus dem Jahr 2009 und hatte noch nicht die AMT-Überwachungschips eingebaut, die inzwischen in allen neuen Computern zu finden waren.
Durch diese Chips konnten Geheimdienste über das Internet unbemerkt Daten mitlesen, den Computer fernsteuern und über die Kamera und das Mikrofon die Umgebung überwachen.
Einige Computer konnten sogar über das Netz heimlich eingeschaltet werden.

Oskar, einer ihrer Freunde, hatte in der Firma, in der er arbeitete, zehn dieser Computer entdeckt.
Niemand dort schien davon zu wissen, sie waren in den Lagerlisten nicht aufgeführt, und so hatte er sie einfach mitgenommen.
Er war mit einem Gabelstapler ins Lager gefahren und hatte eine Menge alter Kartons zusammen mit den Computern aufgeladen und war einfach damit herausgefahren.
Dem Lagerleiter hatte er gesagt, dass er die Kartons für ein Schülerprojekt brauchen würde.
Was für ein Geschenk in Zeiten, in denen die Seriennummer jedes Computers von der Produktion bis zur Müllhalde verfolgt und zusammen mit den E-Mail-Adressen, Telefonnummern, Aufenthaltsorten und Fotos der jeweiligen Besitzer abgespeichert wurde.

Sie steckte einen USB-Stick mit der Aufschrift \enquote{Tails 4.5} in ihr Notebook und drückte den Anschaltknopf.
Tails war das eines der sichersten Betriebssysteme.
Es hinterließ keine bleibenden Spuren auf dem Rechner, auf dem es lief.
Keine Hinweise auf den Benutzer, keine Informationen über seinen Aufenthaltsort.
Wenn man den Stick wieder abzog, blieb nichts von dem übrig, was auf dem Computer gemacht worden war.
Manche Menschen vertrauten sogar mit ihren Leben darauf, dass das sicher funktionierte.

Ihre Knie zitterten leicht.
Tails sollte ihr dabei helfen, das zu tun, was sie jetzt vorhatte.
Auf dem Heimweg von der Schule hatte sie sich entschlossen, nicht noch einen weiteren Tag zu warten.
Sie gab ein langes Passwort ein\footnote{Wo wird in Tails ein langes Passwort eingegeben? ... Um den Persistent Storage freizuschalten. Das ist eine Anspielung auf die Szene in Citizen Four, wo Glenn Greenwald an der gleichen Stelle ein kurzes Passwort eingibt und Edward Snowden die Nase rümpft} und öffnete mit einem Doppelklick das Terminal-Programm.
Manche nannten es Kommando-Zeile: Man gab ein Kommando ein und der Computer antwortete mit Text, keine Bilder, kein Sound, nur Text.
Das war alles.
Aber die Kommandozeile hatte es in sich.
Alle Hacker, die sie kannte, arbeiteten fast ausschließlich damit.
Es gab mächtige Befehle für alles, was man machen wollte.
Und für das was sie vorhatte, waren Maus und Fenster viel zu langsam.
Im Terminal ging alles viel schneller und genauer.

Und das war jetzt wichtig: schnell sein, genau sein.
% chapter 2019_marianne (end)

		
		%== [big-number]#2064# TRON-Attacke
\addchap{2064 -- \textsc{Osiris}-Attacke} % (fold)
\label{cha:2064_osiris_attacke}



\textsc{Lasse und Sigur betraten} ihren Spiele-Raum.\footnote{Änderungen der Anführungszeichen markiere ich mehr nicht gesondert. Auch nicht mehr kleine Tipp- oder Rechtschreibfehler}
\textcolor{gray}{Die Spiele-Räume waren groß und rund, mit Pflanzen, vielen gemalten Fantasy-Bildern an der Wand und einem Podest mit Getränken und viele Arten von Früchten, Nüssen und Süßigkeiten.}\footnote{In diesem Satz sind  viele Informationen. Sie stehen in einem gewissen Widerspruch zum ersten Kapitel. Auch eine moderne Schule wird vielleicht nicht derart viele Annehmlichkeiten schaffen. Das Essen während des Spiels lenkt nur ab und krümelt die Tastatur voll, oder? Woher stammten und wer malte die Fantasy-Bilder? Was sind Fantasy-Bilder? Auch in einer Schule der Zukunft sind vielleicht Karten der Spielewelt Osiris, Ranglisten, wichtige Regeln odgl.  An der Wand.}

Vier große Computer-Bildschirmscheiben schwebten in der Mitte des Raumes über einem Tisch.
Auf dem Tisch lagen zwei Tastaturen, vier \replaced{fast}{etwa} faustgroße Controller und zwei große Glaskugeln, die nach unten hin offen waren.
Lasse und Sigur setzten sich auf die Stühle und nahmen sich die Tastaturen.

\enquote{9:30 Uhr! Es geht los.}, rief Lasse.
Beide saßen gespannt vor ihren Monitoren und schauten auf die Aktivitätsanzeige.
Sigur nickte entschlossen mit dem Kopf.

Lasse: \enquote{Okay.
Alles ruhig.
Dann wollen wir mal loslegen.} 

Er stieß einen Schrei aus.

Lasse: \enquote{S H I T! Sig, mach die Firewall zu, wir bekommen gerade einen massiven Angriff, quantum parallel.}

Sigur hackte mit seiner rechten Hand schnell vier, fünfmal auf seine Tastatur und fixierte  mit seinen Augen den Bildschirm.


Sigur: \enquote{Done! Ist zu! Wer war das?} 

Lasse zuckt mit den Schultern.
\enquote{Keine Ahnung.}

Sigur: \enquote{WAAAASSSSS???!!!}

Lasse: \enquote{Was??}

Sigur trommelte auf seine Tastatur: \enquote{Shit! \dots\  Shit!}

Lasse: \enquote{Sag schon.} 

\textcolor{gray}{Lasse stürmte zum Rechner von Sigur:} \footnote{passt das Verb? In der Einführung des Raumes hatte ich eher den Eindruck, dass die beiden nebeneinander saßen.}\deleted{Lasse: }\enquote{Was ist los?}

Sigur: \enquote{Die Firewall ist wieder offen! Einfach so \dots\  Und ich bin \dots\  draußen, weg, kann nichts tun.
Nicht mal tippen.
Nichts.} Er schlug mit der Faust auf seine Tastatur.

Lasse: \enquote{Shit!}

Sigur: \enquote{Fuck.}

Lasse: \enquote{Zieh den Stecker!}

Sigur: \enquote{Das dürfen wir nicht.
Gegen die Regeln.}

Lasse: \enquote{Scheiß drauf.
\dots\  Okay.
Gut, gut.
Dann eben nicht.} 

Sigur \replaced{stieß}{schmiss} die Tastatur \replaced{von sich}{weg}, stand auf und \replaced{schlug}{haute} ein paar Mal mit der Faust gegen die Wand.

\enquote{Was war das?}, fragte Lasse.

\enquote{WER war das?}, fragte Sigur immer noch mit dem Gesicht zur Wand.

\replaced{Mit einem leisen/lauten Klacken öffnete sich die Tür.}{Klack! Die Tür öffnete sich.}
Beide drehten sich gleichzeitig zur Tür\replaced{.}{, wo} Lilly \added{lugte} mit einem breiten Grinsen herein\deleted{lugte}: \enquote{Hi, hi! Zu langsam, Jungs\dots}

Lasse: \enquote{Oh nein! Nicht.
Bitte nicht \dots\  Nicht schon wieder \dots} 

Lilly war Lasses kleine Schwester und vor zwei Monaten 12 Jahre alt geworden.
Sie spielte \textsc{Osiris} erst seit einem halben Jahr, aber sie hatte wirklich Talent dafür.
\deleted{Sie hatte beim ersten Spiel schon die Dokumente gefunden, die Edward Snowden vom größten amerikanischen Geheimdienst, der NSA, mitgenommen hatte und die damals weltweit für Aufsehen gesorgt hatten.}
\added{Bei ihrem ersten Spiel hatte sie schon die Dokumente gefunden, die Edward Snowden von der NSA, dem größten amerikanischen Geheimdienst, mitgenommen hatte. Die Snowden-Dokumente hatten damals weltweit für großes Aufsehen gesorgt.}\footnote{immer noch zu viel \enquote{hatte}}
\deleted{\textsc{Osiris} hatte daraufhin sofort einige  Missionen hinzufügt, um Lilly zu fangen, und viele Spieler nahmen diese Missionen an.}
\added{Sofort wurden in \textsc{Osiris} einige \added{neue} Missionen hinzufügt, um Lilly zu fangen, und viele Spieler nahmen die Jagd auf Lilly auf.}
Aber sie \replaced{schaffte es}{hatte es geschafft}, zu überleben, und \replaced{niemandem war es gelungen, ihr die Snowden-Dokumente wieder abzujagen}{die Snowden-Dokumente bis zum Spielende zu behalten}.

% chapter 2064_osiris_attacke (end)
		
		%== [big-number]#2064# Schwachstellen in Computern
\addchap{Schwachstellen in Computern} % (fold)
\label{cha:schwachstellen_in_computern}

%[text-caps]#Lasse und Sigur saßen# 
\textsc{Lasse und Sigur saßen} am Esstisch der Hausküche und \textcolor{gray}{hielten sich beide an einer große Tasse mit dickem Kakao fest}.\footnote{beide an einer(sic!) Tasse?}
Lilly saß im Schneidersitz auf dem Tisch und lächelte immer noch.

\enquote{Das hast du vorbereitet}, schnautzte Lasse sie an.
\enquote{Das ging zu schnell.
Das war ein Computer-Programm, oder?}

Lilly: \enquote{Klar! Da habe ich den ganzen Abend gestern dran geschrieben.
Heute morgen habe nur einen Befehl eingegeben und butzzz \dots\ }

Sigur: \enquote{Aber wie ging das? Ich hatte alles dicht.
Ich bin gestern und heute morgen noch einmal alles durchgegangen.
Es war keine Lücke da.
Wo bist du durchgekommen?}

Lilly: \enquote{Eine noch unbekannte Schwachstelle in eurer Firewall.
Die hat ein großer amerikanischer Geheimdienst einbauen lassen und dafür viel Geld bezahlt.
Sie funktioniert super.
Ihr könnt euch das ja nach dem Spiel in den Logdateien anschauen, da steht alles drin.
Erst einmal gehe ich damit noch ein wenig auf Raubzug.
Über diese Hintertür kann man in viele Computer hereinkommen.
Sie ist weit verbreitet und niemand kennt sie.} Sie grinste breit.

Schwachstellen oder Hintertüren gab es in allen Computern und Programmen.
Man konnte sie schließen, wenn man wusste wie.
Wenn man sie erst einmal kannte.
Über Schwachstellen konnte man Computer dazu bringen, alle möglichen Dinge zu tun, die die Programmierer und Computer\replaced{admins}{verwalter} nicht vorgesehen hatten.
\deleted{Manchmal ganz verrückte Dinge.}
Manchmal konnte man einen Computer aus der Ferne komplett übernehmen, als wäre es der eigene, und dann alles mit ihm machen, was man wollte.

\enquote{Das ist doch unfair!}, murrte Sigur.
\enquote{Du kennst eine Hintertür und sagst sie niemandem.
Geheime Hintertüren sollte es im Spiel nicht geben.
Das ist doof.
Wie soll man sich dagegen schützen? Ich habe mir echt viel Arbeit gemacht, es war keine Lücke mehr da.}

Lasse: \enquote{Ich glaube, es ist gar nicht erlaubt, Schwachstellen für sich zu behalten.
Das ist gegen die Spielregeln.
Schwachstellen gehören der Allgemeinheit wie Wasser und Luft.
Man muss sie sofort bekannt machen.
Das ist ein Menschenrecht \dots\  habe ich in meinem \textcolor{gray}{Computerverwalterkurs}\footnote{Admin oder Adminstrator finde ich besser.} gelernt.
Außerdem ist das auch eine Cypherpunk-Regel.} Man kann sogar sein Arbeitsrecht verlieren, wenn man eine Schwachstelle nicht bekannt macht.

Lilly: \enquote{Heute \dots\  heute ist das so.
Damals war es erlaubt, das geheim zu halten! Glaube ich zumindest.
Auf jeden Fall war es Gang und Gäbe.
Ich habe eine E-Mail gelesen, wo jemand eine noch unbekannte Schwachstelle einem anderen zum Verkauf angeboten hat.
Für wahnsinnig viel Geld.
Und der andere hat gekauft.}

\enquote{Blödsinn!} Sigur schüttelte den Kopf.
\enquote{Schwachstellen verkaufen?! Vielleicht wenn du die Mafia spielst, aber sonst nicht.
Das wäre ja total korrupt.
Super schlecht.
\textcolor{gray}{Das ist so schlecht, wie Leute krank machen und dann an der Medizin verdienen.}}\footnote{Vielleicht schärfer: Das ist so schlecht, wie kranken Leuten verdünnte/gepanschte Medizin zu verkaufen.}

Lilly zog die Augenbrauen hoch.
\enquote{Pfff.
Hej, wacht auf.
Die Welt damals war anders.
Das haben alle gemacht damals.
Regierungen, Unternehmen, Geheimdienste.
Die haben ganze Schwachstellen-Arsenale aufgebaut, genauso wie Waffen-Arsenale.
Sie haben sie gehortet.
Massenweise.}

Beide Jungs prusteten los.

\enquote{Lilly, das ist jetzt wirklich Quatsch}, meinte Lasse immer noch lachend.
\enquote{Wer hat dir \added{denn} so etwas erzählt?}

Lilly verschränkte ihre Arme und machte ein ernstes Gesicht: \enquote{Nee.
Echt! Massenweise, Berge \dots\  Ich sag's euch.
Ich habe \dots\ } Sie stockte.

Lasse: \enquote{Lilly, bleib auf dem Teppich.}

\enquote{Du hast \dots\ ?}, fragte Sigur.
\enquote{Was hast du?} Er rutschte zu ihr herüber.

Lilly schaute zu Boden.

\enquote{Du hast noch mehr Schwachstellen gefunden?}, fragte Sigur und schaute sie intensiv an.

Lilly schaute zurück: \enquote{Pfffff.
Jede Menge.
Unglaubliche Sachen.
Ihr könnt euch nicht vorstellen, was alles geht}, sie grinste.
\enquote{Ich habe eine Datenbank bei der NSA gefunden: \enquote{Zero Day Exploits}, das sind alles Schwachstellen, die noch niemand kennt.
Marie ist dort Agentin, sie hat mir dabei geholfen, da dran zu kommen.
Sie selbst darf ja in der Datenbank alles nachschauen, aber sie darf nichts damit tun, was nicht zu ihrem Auftrag passt, und alles, was sie tut und sagt, wird aufgezeichnet, alles, jeder Tastendruck, jedes Gespräch, jeder Gesichtsausdruck.
Sie nehmen echt alles auf! Von allen Mitarbeitern.
Bescheuert.
Echt Scheiße, dort zu arbeiten.
Aber ich konnte mit ihrer Hilfe die Datenbank dort heraus bekommen! War nicht ganz einfach \dots\  Und jetzt kann ich \dots\  Uuuuuhhhh}, sie schüttelte den Kopf, \enquote{\dots\  so viel machen \dots\  Hi.
Hi.
\dots\  Ich weiß, dass das unfair ist, aber es ist ein Spiel.
Da muss man auch verlieren können.}

Sigur murrte: \enquote{Du arbeitest mit der NSA zusammen? Das sind Gegner.
Das ist gemogelt.
Und Marie gibt dir die Informationen, weil ihr auch normal befreundet seid.}

\enquote{Überhaupt nicht.
Ich habe zwei Computer von Hackerinnen für sie geknackt, die sie für ihre Mission gebraucht hat.
Sie hat die Mission dadurch fast fertig.
Die beiden Hackermädels müssen nur noch ins Gefängnis, und mehr als 10 Jahre bekommen.
Sie sind gerade vor Gericht.
Wenn das durch ist, hat sie es geschafft.
Das war der Deal.
Wir haben halt als Frauen gut zusammen gearbeitet, nicht gemogelt! Man kann ja noch jede Menge andere Dinge machen, viel Psychozeugs: Du kannst erpressen, anwerben, bestechen, einschüchtern, lügen.
Ihr wisst schon, was das für eine Welt damals war? Ich meine, wie wollt ihr einen fliegenden Kampfroboter bekommen, ohne jemand bei der CIA zu haben, der tut, was ihr wollt? Kein Wunder, dass ihr noch keinen habt.
Marie hat schon einen.}

Lasse und Sigur schauten sich an.
Fliegende Kampfroboter.
Das war ihr höchstes Ziel.
Sie kannten niemanden, der es geschafft \textcolor{gray}{hatte}\footnote{ich bin unsicher: hatte oder hätte?}, einen zu übernehmen.
Sigur fasste sich an den Bauch: \enquote{Mir wird schlecht.} Lasse kniff seine Lippen zusammen und schaute zu Lilly rüber: \enquote{Wie kann man jemanden bei der CIA umdrehen? Aus dem Kampfroboter-Verband?}, fragte Lasse.

\enquote{Ich weiß nicht, wie Marie das gemacht hat.
Aber du musst halt rauskriegen wie sie dort ticken.
Ich meine, sie hat ja Zugriff auf alle Kommunikationsdaten: Sie schaut, was er so für E-Mails schreibt und bekommt, mit wem er telefoniert, wen er trifft, was er kauft, ob er Feinde hat, ein Verhältnis, von dem seine Frau nichts wissen darf, Schulden, und dann schaut sie, ob sie etwas findet, mit dem sie ihn erpressen kann.
Sie hat anscheinend etwas gefunden.
Oder vielleicht ist jemand auch einfach richtig unzufrieden in seinem Job.
Dann kannst du ihn möglicherweise anwerben.
Marie ist zum Beispiel unzufrieden, sie hat gerade einen Auftrag bekommen, den sie überhaupt nicht machen will.
Aber sie muss.
Und sie kann nicht einmal weggehen von dort.
Das geht nicht so einfach bei Geheimdiensten.
Du kannst es ja mal versuchen, sie für etwas anzuwerben.
Aber pass extrem auf.
Du darfst dich dort wirklich nicht erwischen lassen.
Sonst ist da ist die Hölle los.
Dann ist sofort der halbe Geheimdienst hinter dir her.
Und zack hast du eine Rakete im Zimmer und siehst deine Ehrenmedaille auf dem Bildschirm.
Wie Lars letztes Mal.}

Sigur: \enquote{Lars hat gemogelt.
Deswegen ist er rausgeflogen.
Erpressung und so gehört nicht zum Spiel.
\textsc{Osiris} nimmt einen aus dem Spiel, wenn man so etwas macht.}

Lilly: \enquote{Tut er nicht.
Marie kann jederzeit auf ihren Kampfroboter zugreifen und schauen, was der gerade macht.
Und das geht nicht ohne Erpressung.
Das gehört klar zum Spiel dazu.
Das heißt \enquote{social engineering}, \textcolor{gray}{soziale Technik}\footnote{Wikipedia sagt: angewandte Sozialwissenschaft oder soziale Manipulation}.}

Sigur schwieg.

Lilly: \enquote{Wenn du das nicht willst, kannst du ja Cypherpunks-Missionen spielen.
Da geht es nicht darum, möglichst viele Computer zu übernehmen oder Hacker zu fangen, sondern darum, dass Dinge ans Licht kommen, dass Ungerechtigkeiten ausgeglichen werden und dass immer mehr Menschen der Unterdrückung entkommen können.
Da musst du eher so denken, wie wir heute denken.
Dann kannst du vielleicht herausfinden, wie die Cypherpunks den Krieg gewonnen haben.
Ich meine, das war schon eine Übermacht damals, die Geheimdienste, die Armeen, Riesenunternehmen.
Aber am Ende haben sie es geschafft.
Ich habe mal überlegt, ob es nicht auch spannend wäre, Cypherpunk zu spielen.
Vielleicht ist das sogar der eigentliche Sinn des ganzen Spiels: dass wir lernen können, wie der Befreiungskampf war.}

Lasse und Sigur schauten sich an.
Lilly wollte ihnen gerade erzählen, was der eigentliche Sinn von \textsc{Osiris} ist, dem Spiel, mit dem sie hunderte Stunden verbracht hatten, alles angeschaut, ausprobiert, gelernt hatten, was ihnen über den Weg gelaufen war.

Lasse: \enquote{Hej, Lilly.
Ich spiele \textsc{Osiris} jetzt seit drei Jahren, und es geht am Ende darum, die Welt zu erobern, die Politik zu bestimmen, den Lauf der Geschichte zu ändern, aber nicht um die Befreiung des Internets.
Ich wollte Julian Assange nicht spielen, um das Überwachungssystem \deleted{von damals} zu brechen, Geheimdienste abzuschaffen, Regierungen transparent zu machen, sondern weil er eine Menge coole Leute um sich herum hat, die jede Menge drauf haben, und weil die halbe Welt hinter ihm her ist.}

Lilly: \enquote{Und, hast du die Rolle bekommen?}

Lasse: \enquote{Nein.
Das weißt du.}

Lilly: \enquote{Du kannst sie nicht bekommen\footnote{was eine Bestätigung meiner Änderung im ersten Kapitel ist: Es ist ihm nicht gestattet worden...}, wenn du den Fragebogen am Anfang nicht so ausfüllst, dass es dir hauptsächlich um die Zukunft des Internets geht.
Julian ist ein Cypherpunk.
Wenn du nur Abenteuer willst, dann bekommst du ihn nicht.}

Sigur fauchte dazwischen: \enquote{Woher weißt du das alles?}

Lilly: \enquote{Ich habe Marianne Lasser gelesen: \enquote{Die Geschichte der Cypherpunks}, bevor ich überhaupt angefangen habe, \textsc{Osiris} zu spielen.
Sie schreibt über 2019 und danach, bis zum Kriegsende.
Wie es damals war, wie sie Hacken gelernt hat und dann Cypherpunk geworden ist.
Und sie schreibt, um was es wirklich ging, wie die Leute bei den Geheimdiensten dachten und bei den Unternehmen und wie die Hacker waren, mit weißen, grauen und schwarzen Hüten, also solche, die nur erforschen und nichts kaputt machen, solche bei denen die Grenzen nicht ganz klar sind, die auf ihren eigenen Vorteil aus sind und solche, die echt klauen und zerstören wollen.
Und dann die Hacks, die sie gemacht haben.
Wow! Echt Wahnsinn.
Da habe ich gedacht, das probiere ich in \textsc{Osiris} aus, was sie da schreibt.
Und das stimmte alles im Spiel.
Ganz genau.
Edward Snowden.
Das war genau so, wie Marianne Lasser darüber geschrieben hat.
Genauso hat sie ihren eigenen Hack beschrieben, wie sie die Datenbank mit den Schwachstellen aus dem Hauptquartier der NSA herausschmuggeln konnte.
Und das habe ich dann zusammen mit Marie gemacht.
Das ging gut.
Man braucht dafür einen NSA-Mitarbeiter, sogar einen bestimmten.
Und so haben wir alle Schwachstellen herausgebracht.}

Sigur: \enquote{Alle? -- Wie viele sind das?}

Lilly: \enquote{Ich weiß nicht.
Es ist ein heilloses Durcheinander.
Ich kann mir nicht vorstellen, dass sie selbst eine Übersicht darüber haben.
Bei diesen Geheimdiensten haben viele Schnarchnasen und Langweiler gearbeitet, wie in allen großen Regierungsbehörden.
Aber insgesamt sind es bestimmt tausende, vielleicht zehntausende Schwachstellen, für alle Computer, die du dir nur denken kannst, auch für Satelliten.
Und eine einzelne Schwachstelle davon ist eben die Hintertür, über die ich gerade eben bei euch hineinspaziert bin.}

Lasse: \enquote{Du kannst uns unsere Computer auch wiedergeben \dots\ }

Lilly: \enquote{Nee.
Erstens hat die schon Marie und zweitens wäre das gemogelt.
Für euch ist jetzt erst einmal Flucht dran.}

% chapter schwachstellen_in_computern (end)

	

	\backmatter
		\begin{appendix}
			\chapter*{Weitere Informationen und Kontakt} % (fold)
\label{cha:mehr_informationen}

% \addcontentsline{toc}{chapter}{Weitere Informationen und Kontakt}


% \noindent Weitere Informationen zum Buch finden Sie auf
%
% \noindent \url{https://verlag.rmf.berlin}\\[12pt]


\noindent Weitere Informationen zum Autor Michael Peter Schmidt finden Sie auf unserer Website \url{https://verlag.rmf.berlin}\\[12pt]

% TODO Informationen zur Grafikerin einfügen

\noindent Für Ihre Anregungen, Fragen und Kritik schreiben Sie bitte eine E-Mail an \url{verlag@rmf.berlin}

\vspace{60pt}

\noindent Dieses Buch wurde gesetzt mit {\XeLaTeX} und {\KOMAScript} -- Open Source Textsatz. Nähere Informationen zu {\LaTeX} und  {\XeLaTeX} unter \\\url{www.dante.de} und zu {\KOMAScript} unter \url{www.komascript.de}.

% Für technische Fragen senden Sie bitte eine E-Mail an \url{texlatex@rmf.berlin}



% chapter mehr_informationen (end)
		\end{appendix}
	% \listofchanges

\end{document}
