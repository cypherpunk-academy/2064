\usepackage[autostyle=true,german=guillemets]{csquotes}
\setdefaultlanguage[spelling = new]{german}
\setotherlanguage[spelling = modern]{russian}

\usepackage{verse}
% Auszeichnung des Autors eines Gedichts (s. Kapitel 07 Gedicht »Der Weg von Tschernobyl«)
\newcommand{\attrib}[1]{\nopagebreak{\raggedleft\footnotesize #1\par}}


\usepackage[right]{eurosym}	% Euro-Geldbeträge setzen (mit Euro-Symbol hinter Zahlenwert)

\addto\captionsgerman{%
  \renewcommand{\figurename}{Bild}%
}
\usepackage[labelformat=empty, font=scriptsize, textfont=rm, justification=centering, margin=12pt, aboveskip=6pt]{caption}

\captionsetup[wrapfigure]{labelformat=empty, font=scriptsize, textfont=rm, justification=centering,	margin=12pt, aboveskip=6pt}

% % gefunden in ftp://ftp.dante.de/tex-archive/info/l2tabu/german/l2tabu.pdf 1.8 Nr. 4
\tolerance 1414	% A parameter that tells TeX how much badness is allowable without error. [number] can range from 0 to 10000, and there are no units.
\hbadness 1414	% A parameter that tells TeX at what point to report badness errors (i.e. underfull and overfull error). [number] ranges from 0 to 10000.
\emergencystretch 0.25em	% \emergencystretch (added at TeX3) is used if TeX can not set the paragraph below the \tolerance badness, but rather than make overfull boxes it tries an extra pass "pretending" that every line has an additional \emergencystretch of stretchable glue, this allows the overall badness to be kept below 1000 and stops TeX "giving up" and putting all stretch into one line. So \emergencystretch does not change the setting of "good" paragraphs, it only changes the setting of paragraphs that would have produced over-full boxes.
\hfuzz 0.3pt	% A parameter that allows hbox's to be overfull by [length] before an overfull error occurs.
\widowpenalty=10000	% Penalty for a broken page, with a single line of a paragraph (called "widow") remaining on the top of the succeeding page.
\vfuzz \hfuzz
\raggedbottom	% Command used for top justified within other environments.

\defaultfontfeatures{Mapping=tex-text}
\setmainfont{EB Garamond}
% % Fettschrift definieren: EB Garamond hat keine Fettschrift, die Adobe Garamond Pro keine kyrillischen Glyphen
\setmainfont[BoldFeatures = {Scale=MatchLowercase},
             BoldFont = Adobe Garamond Pro Bold,
             BoldItalicFeatures = {Scale=MatchLowercase},
             BoldItalicFont = Adobe Garamond Pro Bold Italic,
             Ligatures = {TeX,Common}]
            {EB Garamond}
% \setsansfont{Source Sans Pro}
% \setmonofont[Scale=0.9]{Source Sans Pro Light}
\setsansfont{Optima Regular}
\setmonofont[Scale=0.9]{Optima Regular}

\newfontfamily\kolumnenfont[Scale=0.9]{Optima Regular}
\renewcommand*{\headfont}{\kolumnenfont}
