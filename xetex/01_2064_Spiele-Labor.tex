\chapter{2064 -- Das \deleted{Spiele-}Labor} % (fold)
\label{cha:2064_das_labor}


Lasse schlenderte froh auf das \deleted{Spiele-}Labor zu, das in der Mitte des Schulgeländes stand\deleted[remark={Das mit dem Bach sollte später beschrieben werden, wenn es um den konkreten Zugang zum Labor geht. Der Leser muss noch nicht gleich wissen, dass es um ein Spiel geht. Labor ist alles: Chemie, Pharmazie, gefährliche Experimente, Tierversuche, ... Dein Leser muss neugierig werden}]{, wo der Bach sich zweigte und auf beiden Seiten um das Labor herum floss}.

Er sah sich um, betrachtete im \replaced{Vorübergehen}{Vorbeigehen} die \deleted[remark={vermeide »völlig«}]{völlig} unterschiedlich gebauten Häuser, aus Lehm, Holz und großen Steinen\added[remark={weitere Materialien?}]{}, mit fantasievoll\added[remark={fantasievoll regt die Fantasie des Lesers recht wenig an. Später greifst du diese Dorfidylle aber nicht mehr auf, oder? Oder in einem späteren Band?}]{} geformten Fenstern und bunten Dächern\added[remark={was soll ich mir unter bunten Dächern vorstellen? Vielfarbige Ziegel, begrünte Dächer? Rote Ziegeldächer, blau glasierte Ziegel?}]{}.\added{Kaum ein Fenster glich dem anderen, rund, eckig, oval, ...}

\replaced{Er mochte es, an ihnen immer wieder ein neues Detail zu entdecken.}{Er mochte es, an ihnen immer wieder ein neues Detail zu entdecken.} 
\replaced{Man}{Und man} konnte den Häusern \deleted{wirklich} ansehen, wer darin arbeitete\added[remark={Eine Gabe die für das Spiel wichtig ist, oder?}]{, wenn man es verstand, ihre Bauweise und die kleinen Details richtig zusammenzufügen}.
Er lächelte.

\added{Bevor er zum Labor ging, besuchte er Alfred.} \replaced{Am Gehege für die Esel wartete er}{Am Eselgehege wartete Alfred} schon und nickte ihm mit seiner langen Schnauze zu.
Lasse zog eine dicke Karotte aus seiner Tasche\added[remark={In was für einer Tasche trägt Lasse eine  dicke(!) Karotte. Aus diesem Satz entsteht kein Bild im Kopf des Lesers}]{} und streckte sie ihm entgegen.
Der schnappte sie vorsichtig aus seiner Hand\added[remark={Der Esel schnappte die Karotte vorsichtig aus seiner Hand [jetzt wird's unklar. Frisst ein Esel eine dicke Karotte in einem Stück? Beißt er ein paar mal ab? Oder verschwindet  er in eine geschützte Ecke, um dort zu fressen? Kurz: wie fressen Esel dicke Karotten?}]{} und drehte sich damit um.

Zur Insel des \deleted{Spiele-}Labors gab es keine Brücke. \replaced{Der Bach, der durch die Siedlung floß, teilte sich hier  und umgab das Labor wie ein schmaler Burggraben. Wer das Labor betreten wollte,}{Um hereinzukommen,} musste man über den Bach springen\added{ können}\replaced{.}{ und }Lasse suchte sich dafür eine Stelle aus, wo er es gerade so \replaced{schaffen konnte}{schaffte}.

Vor dem Haus drückte er seinen Türöffner in der Hosentasche\added[remark={Du führst ein technisches Mittel ein (Türöffner in der Hosentasche), das später nicht mehr vorkommt. Wichtiger scheint mir hier, jetzt dem Leser zu eröffnen, dass es sich um ein Spiele-Labor handelt, dass man z.B. nur betreten kann, wenn die Spieler/innen eine gewisse körperliche Reife haben. Wozu sonst der »Burggraben«?}]{} und die Tür sprang mit einem leisen Klick\added{en} auf.

Es war neun Uhr, eine halbe Stunde vor Beginn des dritten Spieltages.
Er liebte die \replaced[remark={Lass uns ein anderes Wort für TRON finden. Zum einem kommt es später nicht wirklich vor, und ein neuer Name wäre für späteres Marketing/Merchandising auch rechtlich bedeutsam.}]{XXXX}{TRON}-Wochen, ein riesiges Computerspielturnier, bei dem Tausende von Spielern aus vielen Ländern in einer virtuellen Computerwelt mit- und gegeneinander um die Weltherrschaft spielten.

\deleted[remark={Das wird im Buch nach und nach klar werden. Der Leser hat auch noch keine Information, in welcher Zeit Lasse lebt.}]{Und nicht in einer frei erfundenen Welt, sondern in einem ziemlich originalgetreuen Nachbau des Internets der Jahre 2019 bis 2033, eine grandiose Simulation mit unendlich vielen realistischen Details aus dieser Zeit.}

\deleted{Ein Ziel war es, Missionen zu erfüllen, die man bekam: Zum Beispiel }Computer \added{waren} zu erobern \replaced{und}{, sie} unter Kontrolle zu bekommen\added{.} \replaced{Computer}{und zu verteidigen,} in Banken, in Firmen oder auch bei \replaced{Leuten}{irgendjemandem} zu Hause, der \added{irgend}etwas Interessantes machte\added{n}.
\replaced{Wer richtig gut war,}{Man} konnte auch Satelliten, Schiffe oder Flugzeuge \replaced{kapern}{übernehmen}\replaced{ oder }{,} Agenten und Hacker enttarnen.
\deleted[remark=Textumstellung]{Eine sehr beliebte aber auch schwierige Mission war es, bekannte Hacker wie Kevin Mitnick, Adrian Lamo oder Julian Assange zu jagen.}

\replaced{Nicht nur die meisten Punkte bekam, wer geheime Dokumente vor allem von Geheimdiensten, Regierungen oder großen Unternehmen hackte, er konnte diese Informationen teuer an andere Spieler verkaufen.}{Am meisten Punkte gab es, wenn man an geheime Dokumente kam, vor allem wenn sie von Geheimdiensten, Regierungen oder großen Unternehmen kamen.
Die konnte man teuer verkaufen.}

\added[remark={Eine sehr beliebte... ist zu abstrakt. Es wird sonst zu schnell trocken, wie eine Computerspiel-Anleitung (TL;DR) Mache es ein wenig persönlicher.}]{Die Jugendlichen liebten es bekannte Hacker wie Kevin Mitnick, Adrian Lamo oder Julian Assange zu jagen. Es war nur wahnsinnig schwer diese Missionen zu schaffen.}


Lasse hatte unbedingt Julian Assange spielen wollen, aber das wollten wohl zu viele andere auch\added[remark={Das ist keine schlüssige Begründung: das wollten zu viele andere auch. Es ist ihm mangels Fähigkeiten doch eher bisher nicht gelungen, oder?}]{.}
\replaced{Er hatte sich mit der Rolle eines weniger bekannten Hackers abfinden müssen. Immerhin hatte er es geschafft, }{ und so war er ein weniger bekannter Hacker geworden, mit der Mission} Satelliten, fliegende Kampfroboter und andere Waffen in einem Kriegsgebiet zu übernehmen.

Man musste dafür einiges über \added[remark={Hier taucht das schon beschriebene Zeitproblem auf. In welchem Jahr spielt der Anfang der Geschichte?}]{alte} Computersysteme wissen, aber auch die politische Lage von \replaced{vor 50 Jahren}{damals} kennen und wie die Menschen \added{damals} dachten und fühlten.

\replaced[remark={Damals herrschte Krieg im Internet und in vielen Teilen der Welt. Die Welt war voller schrecklicher Konflikte in Regierungen und in Unternehmen, selbst in den Schulen und vielen Familien.}]{Damals herrschte Krieg im Internet und in vielen Teilen der Welt. Die Welt war voller schrecklicher Konflikte in Regierungen und in Unternehmen, selbst in den Schulen und vielen Familien.}{Damals herrschte Krieg im Internet und auch sonst in der Welt.
Auch in Regierungen und in Unternehmen, und oft auch in Schulen und Familien.}
Es war eine völlig andere Welt.

Lasse betrat das Spielzimmer\added[remark={ie soll sich der Leser das Labor vorstellen? Ein Haus mit mehreren Räumen? Gibt es Laborräume für verschiedene Teams? Gibt es nur einen Raum? Bin ich gleich im «Spielzimmer», wenn ich das Labor betrete? Wie hieße er dann? «Spielzimmer» ist zu kindlich.}]{}, in dem Sigur schon gebannt vor seinem Rechner saß und ab und zu etwas tippte.

Lasse: »Hej, moin!«

Er schlug ihm im Vorbeigehen mit der Hand auf die Schulter und ließ sich in seinen Stuhl fallen.


Sigur war sein Flügelmann\added[remark={Flügelmann assoziiert zwei Flügel, linker und rechter Flügel (Fußball), wer ist dann der andere. Findest du eine andere Bezeichnung? Wolltest du ein hierarchisches Verhältnis zwischen den beiden andeuten?}]{} beim Spiel, sein Partner in der aktuellen Mission.
Er war auch sein Freund, wenn auch nicht der beste.

»Er macht zu oft sein eigenes Ding«, dachte Lasse, »hat zu genaue Vorstellungen, wie die Dinge zu sein hätten.«
Das hatte er ihm  schon oft gesagt.

Aber als Flügelmann war Sigur eine tolle Sache\added[remark={Menschen sind keine Sache. War ein toller Kumpel oder suche was besseres}]{}\deleted{, weil er mit seinen 14 Jahren wirklich viel konnte.
Er hatte im Spiel schon jede Menge Computer in Banken und Ölfirmen übernommen, einmal ein ganzes Passagierflugzeug, das aber dann abgestürzt war.
Sogar in Militärcomputern war er schon unterwegs gewesen.}

\added[remark=Textumstellung]{Mit seinen 14 Jahren hatte er im Spiel schon jede Menge Computer in Banken und Ölfirmen übernommen. Selbst in  besonders schwer zu knackenden Militärcomputern war er schon eingedrungen. Einmal war es ihm gelungen, ein Passagierflugzeug zu übernehmen, das dann aber abgestürzt war.}



Lasse: »Was geht, Sig?«

Sigur reagierte nicht und tippte weiter.


Lasse lugte zu ihm herüber: »Ah! Du bist an der Firewall.
Was ist das Problem?«


Nach einiger Zeit sagte\added[remark={Vermeide sagte. Nur in Ausnahmefällen. Zu unlebendig, gerade auf den ersten Seiten. Murmelte, stotterte,  wie wird etwas gesagt, an eine Person gerichtet, vor sich hin,…  erzeuge eine Stimmung,}]{} Sigur ohne vom Bildschirm wegzuschauen: »Ich weiß nicht.
Nur so ein Gefühl.
Irgendetwas stimmt nicht.«



Die Firewall war das Programm auf jedem Computer, das ungebetene Besucher aus dem Internet abhalten sollte.
Eine Art von Filter- oder Wächterprogramm.\added[remark={Hier taucht ein grundsätzliches Problem auf. Die Story wird zum Sachtext. Das ist langweilig. Der ganze Abschnitt über die Firewall gehört in einen Dialog, dass der Leser ahnt, was eine Firewall macht. Du machst das sehr gut im Dialog. Es ist auch für das erste Kapitel unschädlich, noch nicht zu wissen, was eine Firewall ist. Könnte auch ins Glossar, das wir angedacht hatten.}]{}
Jeder Informationsaustausch mit dem Internet geht in beide Richtungen: raus und rein.
Und auf dem Weg rein kann alles Mögliche mit hineinkommen, was Hacker oder Programmierer zu den normalen Daten hinzugefügt haben, was sich dann im Computer festsetzen und für Verwirrung sorgen kann, oder für Schlimmeres.
So etwas soll die Firewall herausfinden und unschädlich machen.
Und wenn sich etwas im Computer festgesetzt hat, auch dafür sorgen, dass es keine Daten wieder herausschicken kann.

Sigur drehte sich zu Lasse um: »Es waren ein paar seltsame Angriffe von einem Yllil-Computer, aber eigentlich nichts Besonders.

Der hat versucht, irgendwo\added[remark={irgendwo? Nein, konkret wo. Eine Firewall die alles blockt ist nicht irgendwo, sondern wo?}]{} hineinzukommen, aber die Firewall hat alles geblockt.
Fühlt sich trotzdem komisch an…«

Lasse: »Yllil? Das klingt Afrikanisch.
– Bei uns stehen heute aber die Chinesen auf dem Plan!« Er grinste.
»Ich habe super Logdateien von einem Angriff auf einen chinesischen Satelliten gefunden, der fast geklappt hätte.
Da finden wir bestimmt was drin.


Logdateien waren Computerdateien, in denen man alles nachlesen konnte, was auf einem Computer so passiert, warum etwas schief gegangen ist, wie etwas geklappt hat und so weiter.\added[remark={Hier wieder zu trockener Sachtext.}]{}

Für Lasse und Sigur waren Logdateien so etwas wie Zeitungen, nur dass darin genau beschrieben stand, was irgendwo\added[remark={Wieder Irgendwo: Wo genau?}]{} passiert \replaced{war}{ist}\deleted[remark={der Vergleich Log-Datei <-> Zeitung ist ok,  muss aber nicht noch vertieft werden}]{ und nicht nur die Meinung eines anderen Menschen darüber}.

Da stand die genaue Zeit, das Programm, was etwas gemacht hatte und was genau passiert war.\added[remark={Das geht sprachlich besser}]{}

\added[remark={Ich habe verstanden, dass dir diese Theaterdialoge wichtig sind. Aber ich finde das eher störend. Ich möchte als Leser im ersten Kapitel in eine Geschichte hineingezogen werden. Die Theaterdialoge für die Zeit um 1913 sind gut, aber hier habe ich meine Zweifel. Lasse und Sigur sind für mein inneres Auge noch nicht lebendig genug, als dass ich mich auf einen Theater-/Drehbuch-Dialog einlassen könnte.
Die Figuren bewegen sich noch nicht richtig. Sie haben noch keine äußere Gestalt. Ich hoffe, du verstehst mich. Auch die Gemeinschaft und zugleich die Spannung zwischen Lasse und Sigur bleibt so farblos}]{} 
Lasse: »\replaced{Auch}{Noch} einen Kakao\added[remark={wo gibt’s im Labor heiße Getränke?}]{} vorher? Heute holen wir das Ding runter!«

Sigur: »Nicht runter! Übernehmen, Daten kopieren, beobachten, unerkannt bleiben, so lange wie möglich.
Das ist unsere Mission.«

Lasse: »Ja, ist gut… Ich weiß \added{doch}.
Die Mission.
Aber schade.
Ich würde so gerne wissen, was TRON macht, wenn wir den Satelliten tatsächlich abstürzen lassen würden.
Das würde mindestens eine politische Krise geben, Vertuschungsversuche, eine Presseschlacht, ein Meer von Lügen, dann Veschwörungstheorien.\added[remark={Sprechen so 14-jährige Schüler?}]{}
Und dann jede Menge neue Missionen, um uns zu schnappen.
\deleted[remark={Beschreiben: grinste über beide Ohren, lächelte unschuldig als könnte er kein Wässerchen trüben,...}]{He he.}«

Sigur schaute ihn streng an: »Ruhig, \added[remark={wie reden sich Jugendliche an?}]{Alter,} ruhig! Wir wollen das Spiel gewinnen, nicht in fünf Minuten rausfliegen.«

% chapter 2064_das_labor (end)