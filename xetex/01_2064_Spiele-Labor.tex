\addchap{2064 -- Das Labor} % (fold)
\label{cha:2064_das_labor}

\textcolor{gray}{Im Jahr 2064 \dots}% .

Lasse schlenderte froh auf das Labor zu, das in der Mitte des Schulgeländes stand. 
Ein Bach schlängelte sich zwischen den aus Lehm, Holz, groben Steinen, Stroh und Glas gebauten Häusern und den großen, vereinzelt stehenden Bäumen. Kaum ein Haus glich dem anderen.
Die Ziegel der Dächer spielten in zinnoberroten, blau glasierten und dunkelgrünen Farben.
Fenster gab es in allen möglichen geometrischen Formen: Dreiecke, Quadrate, Polygone, Kreise, Ovale.
Alles war bunt, und doch formte es sich zu einem harmonischen Ganzen.
Immer wieder entdeckte er neue Details.
Er lächelte.

Bevor er zum Labor ging, machte er einen Abstecher zu Alfred.
Der wartete schon am Zaun des Eselgeheges und nickte ihm mit seiner langen Schnauze zu.
Lasse zog eine dicke Karotte aus seiner Hosentasche und streckte sie ihm entgegen.
Der schnappte sie vorsichtig aus seiner Hand, drehte sich kauend um und trabte in eine geschützte Ecke.
Lasse schüttelte lächelnd den Kopf.

Das Labor stand zwischen zwei großen Bäumen auf einer Insel, die vom Bach umschlossen war.
Es war ein großer runder Bau mit einer flachen Kuppel und nach außen geschwungenen Wänden, die auf viele runde Zimmer in seinem Inneren hindeuteten.
Es gab keine Brücke.
Um auf die Insel zu kommen, musste Lasse über den Bach springen.
Er suchte sich dafür eine Stelle aus, wo er es gerade so schaffte.

Das Labor war der Ort, an dem die Schüler mit künstlich geschaffenen Welten experimentieren und spielen konnten.
Sie konnten eine Landschaft mit Bergen, Städten, Wäldern, Straßen, Flüssen und einem Stausee schaffen und dann ausprobieren welche Wirkung ein Erdbeben auf die Staumauer hat.
Sie konnten auch neue Materialien erfinden und ihre Eigenschaften und Reaktionen auf andere Stoffe testen.
Expeditionen ins All wurden hier ausgerüstet und ferne Galaxien erforscht.
Wer in die Geschichte zurückzureisen wollte, konnte verschiedene Zeiten wählen, dort in einen Menschen schlüpfen und hautnah erleben, was damals geschah.
Die Computer-Simulationen waren so realistisch, dass einige nur älteren Schülern zugänglich waren.

Lasse lief auf den Eingang des Labors zu. Das Hologramm eines Goblins erschien. Er baute sich breit vor Lasse auf und verschränkte die Arme.

Goblin, mit dunkler Stimme: \enquote{Code.}

Lasse berührte ein Amulett auf seiner Brust und murmelte etwas.

Goblin: \enquote{Korrekt.}

Der Goblin verschwand und die Türe öffnete sich mit einem leisen Klicken.

Im Gebäude kreuzte Lasse den weiten Innenhof und lief auf einen runden Eingang zu, über dem die Namen \enquote{Lasse und Sigur} leuchteten.
Es war neun Uhr, eine halbe Stunde vor Beginn des dritten Spieltages.
Er liebte die Osiris-Wochen, ein riesiges Computerspiel-Turnier, bei dem Tausende von Spielern aus vielen Ländern in einer virtuellen Welt mit- und gegeneinander um die Weltherrschaft spielten.
Und nicht in einer frei erfundenen Welt, sondern in einem ziemlich originalgetreuen Nachbau des Internets der Jahre 2019 bis 2033.

Dort waren Computer zu erobern und zu verteidigen.
Computer in Banken, großen Firmen oder auch bei Leuten zu Hause, die irgendetwas Interessantes machten.
Wer richtig gut war, konnte auch Satelliten, Schiffe oder Flugzeuge übernehmen oder Agenten und Hacker enttarnen.
Die Jugendlichen liebten es, bekannte Hacker wie Kevin Mitnick, Adrian Lamo oder Julian Assange zu jagen.
Es war nur wahnsinnig schwer diese Missionen zu schaffen.
Sehr begehrt waren geheime Dokumente, vor allem von Geheimdiensten, Regierungen oder großen Unternehmen.
Solche Dokumente konnte man teuer an andere Spieler verkaufen.

Lasse hatte eigentlich Julian Assange spielen wollen, aber die Rolle nicht bekommen, noch nicht.
Er musste sich mit den Missionen eines weniger bekannten Hackers abfinden.
Aber immerhin zählten dort zu seinen Aufgaben, Satelliten, fliegende Kampfroboter und andere gefährliche Waffen in einem Kriegsgebiet zu übernehmen und dann selbst einzusetzen.
Man musste dafür einiges über Computersysteme wissen, aber auch die politische Lage von vor 50 Jahren kennen und wie die Menschen dachten und fühlten.

Damals herrschte Krieg im Internet und in vielen Teilen der Welt.
Überall gab es Gewalt und Konflikte, in Regierungen, in Unternehmen, sogar an Schulen und in Familien.
Es war eine völlig andere Welt.
Sie war Lasse fremd, aber gerade darum auch so interessant.

Er betrat seinen Spielraum, in dem Sigur schon gebannt vor seinem Rechner saß und ab und zu etwas tippte.

Lasse: \enquote{Hej, Alter!}

Er schlug ihm im Vorbeigehen mit der Hand auf die Schulter und ließ sich in seinen Stuhl fallen.

Sigur war sein Flügelmann im Spiel, sein Partner in der aktuellen Mission.
Er war auch sein Freund, wenn auch nicht der beste.
\enquote{Er macht zu oft sein eigenes Ding, zu eng}, dachte Lasse, \enquote{er hat zu genaue Vorstellungen, wie die Dinge zu sein hätten.}
Das hatte er ihm schon oft gesagt.

Aber als Flügelmann war Sigur das Beste, was ihm passieren konnte.
Mit seinen 14 Jahren war er wirklich schon sehr weit gekommen.
Er hatte im Spiel schon jede Menge Computer in Banken und Ölfirmen übernommen.
Selbst in besonders schwer zu knackende Militärcomputer war er eingedrungen.
Einmal war es ihm gelungen, ein Passagierflugzeug zu übernehmen, das dann aber abgestürzt war.

Lasse: \enquote{Sig, was geht?}

Sigur reagierte nicht und tippte weiter.

Lasse lugte zu ihm herüber: \enquote{Ah! Du bist an der Firewall.
Was ist das Problem?}

Nach einiger Zeit murmelte Sigur ohne vom Bildschirm wegzuschauen: \enquote{Ich weiß nicht.
Nur so ein Gefühl.
Irgendetwas stimmt nicht.}

Lasse: \enquote{Ist jemand in unsere Computer eingedrungen?
Haben wir ein Loch in der Firewall?
Ist etwas gestohlen worden?}

Sigur schüttelte den Kopf und drehte sich zu Lasse um: \enquote{Es waren ein paar seltsame Angriffe von einem Yllil-Computer, aber eigentlich nichts Besonders.
Der hat versucht, an unsere Passwort-Dateien zu kommen, aber die Firewall hat alles geblockt.
Fühlt sich trotzdem komisch an\dots}

Lasse: \enquote{Yllil? Das klingt Afrikanisch.
-- Bei uns stehen heute aber die Chinesen auf dem Plan!}
Er grinste.
\enquote{Ich habe super Logdateien von einem Angriff auf einen chinesischen Satelliten gefunden, der fast geklappt hätte.
Da drin ist alles haarklein verzeichnet, was der Hacker, äh, die Hackerin, gemacht hat, um reinzukommen.
Echt schlau.
Eine echte Häckse.} 

Er lächelte.

Lasse: \enquote{Und toll zu lesen.
Da finden wir bestimmt etwas für uns drin.}

Lasse: \enquote{Noch einen Kakao vorher? Heute holen wir das Ding runter!}

Er zeigte auf einen kleinen Getränkeautomaten, der an der Wand hing.

Sigur: \enquote{Nicht runter! Übernehmen, Daten kopieren, beobachten, unerkannt bleiben, so lange wie möglich.
Das ist unsere Mission.}

Lasse: \enquote{Ja, ist gut\dots Ich weiß doch.
Die Mission.
Aber schade.
Ich würde so gerne wissen, was \textsc{Osiris} macht, wenn wir den Satelliten tatsächlich abstürzen lassen würden.
Dann wären wir überall in den Nachrichten und weit oben auf der Fahndungsliste.
Das würde jede Menge neue Missionen geben, um uns zu schnappen.} Er grinste über beide Ohren. \enquote{He, he.}

Sigur schaute ihn streng an: \enquote{Alter, ruhig! Wir wollen das Spiel gewinnen, nicht in fünf Minuten rausfliegen.}

% chapter 2064_das_labor (end)