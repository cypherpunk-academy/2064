\addchap{2064 -- Das Labor} % (fold)
\label{cha:2064_das_labor}

\textcolor{gray}{Im Jahr 2064 \dots}% . [remark={Das sollte LaTeX irgendwie ausgegraut in die Mitte schreiben. Es ist wichtig für die Geschichte hier das Jahr zu haben und ich will es nicht im Text haben, ich dachte der Titel reicht, dort ist das Jahr groß gedruckt.}]

\textsc{Lasse schlenderte froh} auf das Labor zu, das in der Mitte des Schulgeländes stand. 
\replaced{Ein Bach schlängelte sich zwischen großen vereinzelt stehenden Bäumen und den aus Lehm, Holz, großen Steinen, Stroh und Glas gebauten Häusern, Kaum ein Haus glich dem anderen. }{Sein Blick strich über die sanften Hügel, die Häuser, die vielen allein stehenden Bäume und den Bach, der sich dazwischen schlängelte.}

\deleted{Im Vorübergehen betrachtete er die einzelnen Gebäude, die aus Lehm, Holz, großen Steinen, Stroh und Glas gebaut waren und ganz unterschiedlich aussahen. Kaum ein Fenster glich dem anderen.} Die Ziegel der Dächer spielten \replaced{von}{in verschiedenen Farben: meist} zinnoberrot\replaced{ über}{,} blau glasiert und dunkelgrün. \added{Die Fenster waren Ausdruck einer Freude an geometrischen Formen: Dreiecke, Quadrate, Polygone, Kreise, Ovale.} Alles war bunt, und doch formte es sich zu einem harmonischen Ganzen.
\replaced{Immer wieder}{Überall} entdeckte er \deleted{immer wieder} neue Details.
Er lächelte.

\replaced{Oft, b}{B}evor er zum Labor ging, besuchte er \deleted{oft} Alfred.
Der wartete heute schon am Zaun des Eselgeheges und nickte ihm mit seiner langen Schnauze zu.
Lasse zog eine dicke Karotte aus seiner Hosentasche und streckte sie ihm entgegen.
Der schnappte sie vorsichtig aus seiner Hand,\deleted{ nahm sie ganz in sein Maul,} \deleted{und} drehte sich kauend \deleted{damit} um \added{ und trabte in eine geschützte Ecke}.
Lasse schüttelte lächelnd den Kopf.

Das Labor stand zwischen zwei großen Bäumen auf einer Insel, \replaced{die vom Bach umschlossen war.}{um die der Bach an beiden Seiten herumfloss.}
Es war ein großer runder Bau mit einer flachen Kuppel und \added{nach außen} geschwungenen Wänden, die auf viele runde Zimmer \added{in seinem Inneren} hindeuteten.
Es gab keine Brücke.
Um auf die Insel zu kommen, musste Lasse über den Bach springen.
Er suchte sich dafür eine Stelle aus, wo er es gerade so schaffte.

Das Labor war der Ort, an dem eine Landschaft mit Bergen, Städten, Wäldern, Straßen,  Flüssen und einem Stausee geschaffen werden konnte. Welche Wirkung würde ein Erdbeben auf die Staumauer haben?\footnote{Meine kläglichen Versuche, von \enquote{man} wegzukommen}
\deleted{Das Labor war der Ort, an dem man experimentieren konnte: Man konnte ein Landschaft mit Bergen, Städten, Wäldern, Straßen und Flüssen erschaffen, einen Stausee bauen und den dann durch ein Erdbeben brechen lassen.}
Neue Materialien wurden erfunden und in ihren Fähigkeiten und in ihren Reaktionen auf andere Stoffe getestet.
\deleted{Man konnte neue Materialien erfinden und ausprobieren, was man mit ihnen machen konnte, wie sie mit anderen Materialien reagierten.}
Expeditionen ins All wurden ausgerüstet und ferne Galaxien erforscht.\deleted{Oder eine Expedition zu fernen Planeten machen und diese erforschen.}
Wer bereit war, in die Geschichte zurückzureisen, konnte dort in einen Menschen schlüpfen und hautnah frühere Zeiten erleben.\deleted{Man konnte auch in der Geschichte zurückreisen, in alle möglichen Zeiten, und dort verschiedene Rollen annehmen, die Eigenheiten der Zeit kennenlernen und Dinge ändern.}
\deleted{Natürlich ging das alles nur in Computer-Simulationen.
Aber die waren so realistisch,} Diese Computer--Simulationen waren so realistisch, dass einige  nur  älteren Schülerinnen zugänglich waren.

Lasse lief auf den Eingang des Labors zu. Das Hologramm einer Fee\footnote{ich bin gegen die Fee. Das Labor ist ein Ort besonderer Herausforderungen. Wer oder was könnte ein passender Türwächter sein? Ein abstraktes Symbol wie: Erlenmeyerkolben oder eine der Figuren der Akademie?} erschien, die ein wenig größer war als er selbst.

\deleted{Fee: }»Dein Code, Lasse!«

Lasse berührte ein Amulett auf seiner Brust und murmelte etwas.

\deleted{Fee: »Danke.«}

\replaced{»Danke«, quittierte d}{D}ie Fee \added{und }verschwand\replaced{. D}{ und d}ie Türe öffnete sich mit einem leisen Klicken.

Im Gebäude kreuzte er den weiten Innenhof und lief auf einen runden Eingang zu, über dem \replaced{ihre Namen}{die Worte} »Lasse und Sigur« leuchteten.
Es war neun Uhr, eine halbe Stunde vor Beginn des dritten Spieltages.
Er liebte die OSIRIS-Wochen, ein riesiges Computerspiel-Turnier, bei dem Tausende von Spielern aus vielen Ländern in einer virtuellen \replaced{W}{Computerw}elt mit- und gegeneinander um die Weltherrschaft spielten.
Und nicht in einer frei erfundenen Welt, sondern in einem ziemlich originalgetreuen Nachbau des Internets der Jahre 2019 bis 2033.

Dort waren Computer zu erobern und zu verteidigen\replaced{. Computer}{,} in Banken, großen Firmen oder auch bei Leuten zu Hause, die irgendetwas Interessantes machten.
Wer richtig gut war, konnte auch Satelliten, Schiffe oder Flugzeuge übernehmen oder Agenten und Hacker enttarnen.
Die Jugendlichen liebten es, bekannte Hacker wie Kevin Mitnick, Adrian Lamo oder Julian Assange zu jagen.
Es war nur wahnsinnig schwer diese Missionen zu schaffen.
Sehr begehrt waren geheime Dokumente, vor allem von Geheimdiensten, Regierungen oder großen Unternehmen. \replaced{Solche wichtigen Dokumente}{Die} konnte man teuer an andere Spieler verkaufen.

Lasse hatte eigentlich Julian Assange spielen wollen, aber \replaced{das war ihm noch nicht gestattet worden}{das hatte nicht geklappt}.\footnote{Hier ein Wink zur späteren Akademie. Ist auch schlüssiger zum folgenden \enquote{abfinden}}
Er musste sich mit der Rolle eines weniger bekannten Hackers abfinden.
Immerhin zählten zu seinen Aufgaben, Satelliten, fliegende Kampfroboter und andere gefährliche Waffen in einem Kriegsgebiet zu übernehmen und damit Missionen zu erfüllen.

Man musste dafür einiges über Computersysteme wissen, aber auch die politische Lage von vor 50 Jahren kennen und wie die Menschen dachten und fühlten.

Damals herrschte Krieg im Internet und in vielen Teilen der Welt.
Überall gab es Gewalt und Konflikte, in Regierungen, in Unternehmen, sogar an Schulen und in Familien.
Es war eine völlig andere Welt.
Sie war Lasse fremd, aber gerade darum auch \added{so} interessant.

Lasse betrat seinen Spielraum, in dem Sigur schon gebannt vor seinem Rechner saß und ab und zu etwas tippte.

Lasse: »Hej, Alter!«

Er schlug ihm im Vorbeigehen mit der Hand auf die Schulter und ließ sich in seinen Stuhl fallen.

Sigur war sein Flügelmann im Spiel, sein Partner in der aktuellen Mission. %\footnote{WikiPedia: In Luftstreitkräften wird das Wort auch heute noch als taktischer Begriff benutzt, jedoch meist in der englischen Version Wingman. Ein Wingman ist ein Pilot, der einen anderen Piloten in einem feindlichen Umfeld unterstützt.}
Er war auch sein Freund, wenn auch nicht der beste.
»Er macht zu oft sein eigenes Ding«, dachte Lasse, »hat zu genaue Vorstellungen, wie die Dinge zu sein hätten.«
Das hatte er ihm  schon oft gesagt.
%\footnote{Sigur ist hier in seinem Charakter sehr genau und penibel, das führt dann später dazu, dass er über die Grenzen hinaus geht. Lasse probiert die Grenzen aus und kennt sich deswegen damit besser aus. \enquote{Er macht zu oft sein eigenes Ding}ist nicht sauber ausgedrückt, Lasse meint, er drückt zu oft das durch was er will, aber ich will diesen Fehler gerne drin lassen, weil Lasse auch erst 14 ist.}

Aber als Flügelmann war Sigur das Beste, was ihm passieren konnte\replaced{. M}{, weil er m}it seinen 14 Jahren \replaced{war er wirklich schon sehr weit gekommen.}{wirklich viel konnte}.
Er hatte im Spiel schon jede Menge Computer in Banken und Ölfirmen übernommen. Selbst in besonders schwer zu knackende Militärcomputer war er \deleted{schon} eingedrungen.
Einmal war es ihm gelungen, ein Passagierflugzeug zu übernehmen, das dann aber abgestürzt war.

Lasse: »Sig, was geht?«

Sigur reagierte nicht und tippte weiter.

Lasse lugte zu ihm herüber: »Ah! Du bist an der Firewall.
Was ist das Problem?«

Nach einiger Zeit murmelte Sigur ohne vom Bildschirm wegzuschauen: »Ich weiß nicht.
Nur so ein Gefühl.
Irgendetwas stimmt nicht.«

Lasse: »Ist jemand in unsere Computer eingedrungen?
Haben wir ein Loch in der Firewall?
Ist etwas gestohlen worden?«

Sigur schüttelte den Kopf und drehte sich zu Lasse um: »Es waren ein paar seltsame Angriffe von einem Yllil-Computer, aber eigentlich nichts Besonders.
Der hat versucht, an unsere Passwort-Dateien zu kommen, aber die Firewall hat alles geblockt.
Fühlt sich trotzdem komisch an\dots«

Lasse: »Yllil? Das klingt Afrikanisch.
-- Bei uns stehen heute aber die Chinesen auf dem Plan!« Er grinste.
»Ich habe super Logdateien von einem Angriff auf einen chinesischen Satelliten gefunden, der fast geklappt hätte.
Da drin ist alles haarklein verzeichnet, was der Hacker, äh, die Hackerin, gemacht hat, um \replaced{r}{hin}einzukommen.
Echt schlau.
Eine echte Häckse.« 

Er lächelte.

Lasse: »Und toll zu lesen.
Da finden wir bestimmt etwas für uns drin.«

%[remark={Warum der Kommentar hier? Der erste Theater-Dialog kommt im 8. Kapitel: Ich habe verstanden, dass dir diese Theaterdialoge wichtig sind. Aber ich finde das eher störend. Ich möchte als Leser im ersten Kapitel in eine Geschichte hineingezogen werden. Die Theaterdialoge für die Zeit um 1913 sind gut, aber hier habe ich meine Zweifel. Lasse und Sigur sind für mein inneres Auge noch nicht lebendig genug, als dass ich mich auf einen Theater-/Drehbuch-Dialog einlassen könnte.
%Die Figuren bewegen sich noch nicht richtig. Sie haben noch keine äußere Gestalt. Ich hoffe, du verstehst mich. Auch die Gemeinschaft und zugleich die Spannung zwischen Lasse und Sigur bleibt so farblos}]

Lasse: »Noch einen Kakao vorher?« Er zeigte auf den kleinen Getränkeautomaten, der an der Wand hing. »Heute holen wir das Ding runter!«

Sigur: »Nicht runter! Übernehmen, Daten kopieren, beobachten, unerkannt bleiben, so lange wie möglich.
Das ist unsere Mission.«

Lasse: »Ja, ist gut… Ich weiß doch.
Die Mission.
Aber schade.
Ich würde so gerne wissen, was OSIRIS macht, wenn wir den Satelliten tatsächlich abstürzen lassen würden.
Dann wären wir überall in den Nachrichten und weit oben auf der Fahndungsliste.
Das würde jede Menge neue Missionen geben, um uns zu schnappen.«

Er grinste über beide Ohren: »He, he.«

Sigur schaute ihn streng an: »Alter, ruhig! Wir wollen das Spiel gewinnen, nicht in fünf Minuten rausfliegen.«

% chapter 2064_das_labor (end)