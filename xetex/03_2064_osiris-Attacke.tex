%== [big-number]#2064# TRON-Attacke
\addchap{2064 -- \textsc{Osiris}-Attacke} % (fold)
\label{cha:2064_osiris_attacke}



\textsc{Lasse und Sigur betraten} ihren Spiele-Raum.\footnote{Änderungen der Anführungszeichen markiere ich mehr nicht gesondert. Auch nicht mehr kleine Tipp- oder Rechtschreibfehler}
\textcolor{gray}{Die Spiele-Räume waren groß und rund, mit Pflanzen, vielen gemalten Fantasy-Bildern an der Wand und einem Podest mit Getränken und viele Arten von Früchten, Nüssen und Süßigkeiten.}\footnote{In diesem Satz sind  viele Informationen. Sie stehen in einem gewissen Widerspruch zum ersten Kapitel. Auch eine moderne Schule wird vielleicht nicht derart viele Annehmlichkeiten schaffen. Das Essen während des Spiels lenkt nur ab und krümelt die Tastatur voll, oder? Woher stammten und wer malte die Fantasy-Bilder? Was sind Fantasy-Bilder? Auch in einer Schule der Zukunft sind vielleicht Karten der Spielewelt Osiris, Ranglisten, wichtige Regeln odgl.  An der Wand.}

Vier große Computer-Bildschirmscheiben schwebten in der Mitte des Raumes über einem Tisch.
Auf dem Tisch lagen zwei Tastaturen, vier \replaced{fast}{etwa} faustgroße Controller und zwei große Glaskugeln, die nach unten hin offen waren.
Lasse und Sigur setzten sich auf die Stühle und nahmen sich die Tastaturen.

\enquote{9:30 Uhr! Es geht los.}, rief Lasse.
Beide saßen gespannt vor ihren Monitoren und schauten auf die Aktivitätsanzeige.
Sigur nickte entschlossen mit dem Kopf.

Lasse: \enquote{Okay.
Alles ruhig.
Dann wollen wir mal loslegen.} 

Er stieß einen Schrei aus.

Lasse: \enquote{S H I T! Sig, mach die Firewall zu, wir bekommen gerade einen massiven Angriff, quantum parallel.}

Sigur hackte mit seiner rechten Hand schnell vier, fünfmal auf seine Tastatur und fixierte  mit seinen Augen den Bildschirm.


Sigur: \enquote{Done! Ist zu! Wer war das?} 

Lasse zuckt mit den Schultern.
\enquote{Keine Ahnung.}

Sigur: \enquote{WAAAASSSSS???!!!}

Lasse: \enquote{Was??}

Sigur trommelte auf seine Tastatur: \enquote{Shit! \dots\  Shit!}

Lasse: \enquote{Sag schon.} 

\textcolor{gray}{Lasse stürmte zum Rechner von Sigur:} \footnote{passt das Verb? In der Einführung des Raumes hatte ich eher den Eindruck, dass die beiden nebeneinander saßen.}\deleted{Lasse: }\enquote{Was ist los?}

Sigur: \enquote{Die Firewall ist wieder offen! Einfach so \dots\  Und ich bin \dots\  draußen, weg, kann nichts tun.
Nicht mal tippen.
Nichts.} Er schlug mit der Faust auf seine Tastatur.

Lasse: \enquote{Shit!}

Sigur: \enquote{Fuck.}

Lasse: \enquote{Zieh den Stecker!}

Sigur: \enquote{Das dürfen wir nicht.
Gegen die Regeln.}

Lasse: \enquote{Scheiß drauf.
\dots\  Okay.
Gut, gut.
Dann eben nicht.} 

Sigur \replaced{stieß}{schmiss} die Tastatur \replaced{von sich}{weg}, stand auf und \replaced{schlug}{haute} ein paar Mal mit der Faust gegen die Wand.

\enquote{Was war das?}, fragte Lasse.

\enquote{WER war das?}, fragte Sigur immer noch mit dem Gesicht zur Wand.

\replaced{Mit einem leisen/lauten Klacken öffnete sich die Tür.}{Klack! Die Tür öffnete sich.}
Beide drehten sich gleichzeitig zur Tür\replaced{.}{, wo} Lilly \added{lugte} mit einem breiten Grinsen herein\deleted{lugte}: \enquote{Hi, hi! Zu langsam, Jungs\dots}

Lasse: \enquote{Oh nein! Nicht.
Bitte nicht \dots\  Nicht schon wieder \dots} 

Lilly war Lasses kleine Schwester und vor zwei Monaten 12 Jahre alt geworden.
Sie spielte \textsc{Osiris} erst seit einem halben Jahr, aber sie hatte wirklich Talent dafür.
\deleted{Sie hatte beim ersten Spiel schon die Dokumente gefunden, die Edward Snowden vom größten amerikanischen Geheimdienst, der NSA, mitgenommen hatte und die damals weltweit für Aufsehen gesorgt hatten.}
\added{Bei ihrem ersten Spiel hatte sie schon die Dokumente gefunden, die Edward Snowden von der NSA, dem größten amerikanischen Geheimdienst, mitgenommen hatte. Die Snowden-Dokumente hatten damals weltweit für großes Aufsehen gesorgt.}\footnote{immer noch zu viel \enquote{hatte}}
\deleted{\textsc{Osiris} hatte daraufhin sofort einige  Missionen hinzufügt, um Lilly zu fangen, und viele Spieler nahmen diese Missionen an.}
\added{Sofort wurden in \textsc{Osiris} einige \added{neue} Missionen hinzufügt, um Lilly zu fangen, und viele Spieler nahmen die Jagd auf Lilly auf.}
Aber sie \replaced{schaffte es}{hatte es geschafft}, zu überleben, und \replaced{niemandem war es gelungen, ihr die Snowden-Dokumente wieder abzujagen}{die Snowden-Dokumente bis zum Spielende zu behalten}.

% chapter 2064_osiris_attacke (end)