% == [big-number]#2019# Marianne
\addchap{2019 -- Marianne} % (fold)
\label{cha:2019_marianne}

\textcolor{gray}{Im Jahr 2019 \dots} %\footnote{mittig, grau}

Marianne gähnte. Deutsch-Unterricht, 10.\,Klasse, gedrückte Langeweile im Raum.
Anni neben ihr chattete unter dem Tisch mit ihrem Smartphone.
Sie fischte eine Brotdose aus ihrer Schultasche und platzierte sie vor sich auf den Tisch.
Sie öffnete sie, holte in Papier eingewickelte Butter, Käse und Gurkenscheibchen heraus und ordnete \replaced{alles}{sie} liebevoll nebeneinander an.
Dann nahm sie ein Messer heraus und wollte gerade anfangen eine Brotscheibe mit Butter zu beschmieren, als der Lehrer plötzlich neben ihr stand.


Lehrer: \enquote{Na? Und warum machst du das nicht zu Hause?} Er wippte mit den Füßen.

Marianne schaute ihn mit ruhigem Blick an.
In ihr stieg eine Wut hoch: \enquote{Das ist im Augenblick das Spannendste und Kreativste, was ich tun kann.}

\deleted{Lehrer spitz: }\enquote{Spannender als Faust? Das kann ich mir kaum vorstellen.
Entweder du\dots}

Marianne unterbrach ihn: \enquote{Ich sitze hier seit über einer Stunde rum und muss mir anhören, was Leute in den letzten 200 Jahren über Faust und Mephisto ausgefurzt haben.
Was soll ich damit?
Ich kenne die Leute nicht einmal.
Was hat das mit meinem Leben zu tun?
Hat das überhaupt mit irgendwas heute zu tun?}

Der Lehrer drehte sich abrupt um, atmete ein paar Mal kräftig durch, zeigte in Richtung Tür und schrie \enquote{Raus!}
Er schloss die Augen.
Marianne packte ihre Sachen zusammen, nahm ihre Tasche und ging am Lehrer vorbei aus dem Klassenraum.

\deleted{Er rief ihr hinterher: }\enquote{Du wartest draußen, direkt vor der Tür.}, \added{rief er ihr hinterher}, \enquote{Wir sprechen uns nach dem Unterricht.}
Die Tür krachte ins Schloss.

Marianne machte sich sofort auf den Heimweg.
\enquote{So ein Weichei}, dachte sie, \enquote{wahrscheinlich war das \enquote{ausgefurzt} zu viel für ihn gewesen.} Immerhin hatte er eines der Bücher über Faust, die sie lesen mussten, selbst geschrieben und so konnte er es durchaus persönlich nehmen.
Sie mochte ihn eigentlich, er war ganz cool, als Mensch, aber in der Lehrerrolle\dots Wahrscheinlich mochte er die selbst nicht.
\enquote{Echt ein Weichei}, dachte sie und schüttelte den Kopf.
\enquote{Und was mache ich jetzt mit dem angefangenen Tag?} \enquote{Klar!}, sagte sie laut und schnippte mit den Fingern.

Eine halbe Stunde später saß sie oben auf der großen Treppe rechts neben dem Eingang zum Rathaus Neukölln in einer schattigen Ecke.
Es war ein heißer Tag, um die 28 Grad, und sie trug jetzt ein kurzes, luftiges Kleid, das sie sich auf einem Sprung nach Hause angezogen hatte.
Sie fühlte sich darin ein wenig unwohl, normalerweise trug sie so etwas nicht.
Aber jetzt erfüllte es seinen Zweck.

Sie zog einen noch eingeschweißten Laptop aus ihrem Stoffbeutel.
Er war unbenutzt, aber nicht mehr auf dem neuesten Stand der Technik.
Er war aus dem Jahr 2009 und hatte noch nicht die Überwachungschips eingebaut, die inzwischen in allen neuen Computern zu finden waren.
\deleted{Durch diese Chips konnten Geheimdienste Computer über das Internet unbemerkt fernsteuern, Daten mitlesen oder über die Kamera und das Mikrofon die Umgebung überwachen. Einige Computer konnte man darüber sogar über das Netz anschalten.}
Durch diese Chips konnten Geheimdienste über das Internet unbemerkt Daten mitlesen, den Computer fernsteuern und über die Kamera und das Mikrofon die Umgebung überwachen. Einige Computer konnten sogar über das Netz heimlich eingeschaltet werden.


Oskar, einer ihrer Freunde, hatte in der Firma, in der er arbeitete, zehn dieser Computer entdeckt.
Niemand dort schien davon zu wissen, sie waren in \replaced[remark={Plural scheint mir flüssiger}]{den Lagerlisten}{der Lagerliste} nicht aufgeführt, und so hatte er sie einfach mitgenommen.
Er war mit einem Gabelstapler ins Lager gefahren und hatte eine Menge alter Kartons zusammen mit den Computern aufgeladen und war einfach damit herausgefahren.
Dem Lagerleiter hatte er gesagt, dass er die Kartons für ein Schülerprojekt brauchen würde.
Was für ein Geschenk in Zeiten, in denen die Seriennummer jedes Computers von der Produktion bis zur Müllhalde verfolgt und zusammen mit den E-Mail-Adressen, Telefonnummern, Aufenthaltsorten und Fotos der jeweiligen Besitzer abgespeichert wurde.

Sie steckte einen USB-Stick mit der Aufschrift \enquote{Tails 4.5} in ihr Notebook und drückte den Anschaltknopf.
Tails war das eines der sichersten Betriebssysteme.
Es hinterließ keine bleibenden Spuren auf dem Rechner, auf dem es \replaced{gestartet worden war}{lief}.
Keine Hinweise auf den Benutzer, keine Informationen über seinen Aufenthaltsort.
Wenn man den Stick wieder abzog, blieben nichts von dem übrig, was auf dem Computer gemacht worden war.
Manche Menschen vertrauten sogar mit ihren Leben darauf, dass das sicher funktionierte.

Ihre Knie zitterten leicht.
Tails sollte ihr dabei helfen, das zu tun, was sie jetzt vorhatte.
Sie hatte sich auf dem Heimweg von der Schule entschlossen, nicht noch einen weiteren Tag zu warten.
Sie gab ein langes Passwort ein\footnote{Wo wird in Tails ein langes Passwort eingegeben?} und öffnete \deleted{dann} mit einem Doppelklick das Terminal-Programm von Tails.
Manche nannten es Kommando-Zeile: Man gab ein Kommando ein und der Computer antwortete mit Text.
Das war alles.
Aber die Kommandozeile hatte es in sich.
Alle Hacker, die sie kannte, arbeiteten fast ausschließlich damit.
Es gab mächtige Befehle für alles, was man machen wollte.
Und für das was sie vorhatte, waren Maus und Fenster viel zu langsam.
Im Terminal ging alles viel schneller und \replaced{präziser}{genauer} als mit allen anderen Programmen.

Und das war jetzt wichtig: schnell sein, genau sein.
% chapter 2019_marianne (end)
