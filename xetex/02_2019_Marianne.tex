% == [big-number]#2019# Marianne
\addchap{2019 -- Marianne} % (fold)
\label{cha:2019_marianne}

% [text-caps]#Marianne gähnte.# Deutsch-Unterricht, 10.
Marianne gähnte. Deutsch-Unterricht, 10.\,Klasse, gedrückte Langeweile im Raum.
Anni neben ihr chattete unter dem Tisch mit ihrem Smartphone.
Sie \deleted{selbst} fischte eine Brotdose aus ihrer Schultasche und platzierte sie vor sich auf dem Tisch.
Sie öffnete sie, holte in Papier eingewickelte Butter, Käse und Gurkenscheibchen heraus und ordnete sie liebevoll nebeneinander an.
Dann nahm sie ein Messer heraus\deleted{, nahm eine Brotscheibe} und wollte gerade anfangen \added{eine Brotscheibe mit }\deleted{, die }Butter \deleted{darauf} zu schmieren, als der Lehrer plötzlich neben ihr stand.


Lehrer: \enquote{Na? Und warum machst du das nicht zu Hause?} Er wippte mit den Füßen.

Marianne schaute ihn mit ruhigem Blick an.
In ihr stieg eine Wut hoch: \enquote{Das ist im Augenblick das Spannendste und Kreativste, was ich tun kann.}

Lehrer spitz: \enquote{Spannender als Faust? Das kann ich mir kaum vorstellen.
Entweder du\dots}

Marianne unterbrach ihn: \enquote{Ich sitze hier seit über einer Stunde rum und muss mir anhören, was Leute in den letzten 200 Jahren über Faust und Mephisto \deleted{…} ausgefurzt haben.
Was soll ich damit? Ich kenne die Leute nicht einmal.
Was hat das mit meinem Leben zu tun? Hat das überhaupt mit irgendwas heute zu tun?} 

Der Lehrer drehte sich abrupt um, atmete ein paar Mal kräftig durch, zeigte in Richtung Tür und schrie \enquote{Raus!} Er schloss die Augen.
Marianne packte ihre Sachen zusammen, nahm ihre Tasche und ging am Lehrer vorbei aus dem \replaced{Klassenraum}{Zimmer}.

Er rief ihr hinterher: \enquote{Du wartest draußen, direkt vor der Tür.
Wir sprechen \replaced{uns noch}{danach}.} Die Tür krachte zu.

Marianne machte sich sofort auf den Heimweg.
\enquote{So ein Weichei}, dachte sie, \enquote{wahrscheinlich war das \enquote{ausgefurzt} zu viel für ihn gewesen.} Immerhin hatte er eines der Bücher über Faust, die sie lesen mussten, selbst geschrieben und so konnte er es durchaus persönlich nehmen.
Sie mochte ihn eigentlich, er war \deleted{so} ganz cool, als Mensch, aber in der Lehrerrolle\replaced{?}{\dots} Wahrscheinlich mochte er die selbst nicht.
\enquote{Echt ein Weichei}, dachte sie und schüttelte den Kopf.
\enquote{Und was mache ich jetzt mit dem angefangenen Tag? \dots\
Klar!}, sagte sie laut und schnippte mit den Fingern.

Eine halbe Stunde später saß sie oben auf der großen Treppe rechts neben dem Eingang \replaced{zum}{des} Rathaus\deleted{es} Neukölln in einer schattigen Ecke.
Es war ein heißer Tag, um die 28 Grad, und sie trug jetzt ein kurzes, luftiges Kleid, das sie sich auf einem Sprung nach Hause angezogen hatte.
Sie fühlte sich darin ein wenig unwohl, normalerweise trug sie so etwas nicht.
Aber jetzt erfüllte es seinen Zweck.

Sie zog einen noch eingeschweißten Laptop aus ihrem Stoffbeutel.
Er war unbenutzt, aber nicht mehr  \replaced[remark={Das erscheint mir sprachlich ungenau: eingeschweißt bedeutet neu, \emph{nicht mehr ganz neu }heißt gebraucht.}]{der neuestes Stand der Technik}{ganz neu}.
Er war aus dem Jahr 2009 und hatte noch nicht die Überwachungschips eingebaut, die inzwischen in allen neuen Computern zu finden waren.
\textcolor{gray}{Mit ihnen konnten Geheimdienste jeden Computer über das Internet fernsteuern, konnten Dateien anschauen, kopieren, löschen, sogar ein neues Betriebssystem installieren, wenn sie wollten, und einige Computer konnte man darüber sogar über das Netz anschalten.}\added[remark={Ich würde das dunkler, nicht so konkret formulieren. Z.\,B.: Durch diese Chips konnten Geheimdienste  Computer unbemerkt fernsteuern, Daten mitlesen oder über die Kamera und das Mikrofon die Umgebung überwachen.}]{}


Oskar, einer ihrer Freunde, hatte in der Firma, in der er arbeitete zehn d\added{ies}er Computer entdeckt.
Niemand dort schien davon zu wissen, sie waren in der Lagerliste nicht aufgeführt, und so hatte er sie einfach mitgenommen.
Er war mit einem Gabelstapler ins Lager gefahren und hatte eine Menge alter Kartons zusammen mit den Computern aufgeladen und war einfach damit herausgefahren.
Dem Lagerleiter hatte er gesagt, dass er die Kartons für ein Schülerprojekt brauchen würde.
Was für ein Geschenk in Zeiten, in denen die Seriennummer jedes Computers von der Produktion bis zur Müllhalde verfolgt und zusammen mit den E-Mail-Adressen, Telefonnummern, Aufenthaltsorten und Fotos der jeweiligen Besitzer abgespeichert wurde.

Sie steckte einen \replaced{USB-S}{Speichers}tick mit der Aufschrift \enquote{Tails \deleted[remark={jetzt Version 3.0 -- wir wissen die Versionssnummer von 2019 nicht}]{2.3}} in \replaced{ihr Notebook}{einen Steckplatz am Computer} und drückte den Anschaltknopf.
\replaced{Tails war das sicherste Betriebssystem, das keine Spuren hinterließ. Keine Hinweise auf den Benutzer, keine Informationen über seinen Aufenthaltsort.}{Tails war ein Programm, oder besser gesagt ein Betriebssystem, das dafür sorgte, dass niemand herausfinden konnte, wer den Computer, auf dem es lief, gerade nutzte.
Und auch wo er auf der Welt gerade war.}
\replaced{Und zog man den Sticke wieder ab,}{Und wenn man den Stick wieder abzog}, \deleted{dann} blieben keine Spuren \added{mehr} von dem, was \replaced{auf dem Computer gemacht worden war.}{man gemacht hatte}.
Manche Menschen vertrauten sogar mit ihren Leben darauf, dass das funktionierte.
\deleted[remark={Das sollte vielleicht eher ins Glossar.}]{Mit Tails konnte man alles machen, was man anonym machen wollte: E-Mails an Journalisten schicken, mit Hackerfreunden chatten, auf überwachten Seiten im Internet surfen oder auch einfach nur einen Artikel für ein Untergrund-Magazin schreiben.}

\replaced{Ihre Knie zitterten leicht. Tails sollte ihr helfen, das zu tun, was sie jetzt vorhatte.}{Marianne interessierte das alles jetzt nicht.
Sie hatte etwas anderes vor.
Und deswegen zitterten ihre Knie leicht.}
Auf dem Heimweg von der Schule hatte sie sich \deleted{dazu} entschlossen, nicht noch \deleted{einmal} einen \added{weiteren} Tag zu warten. \textcolor{gray}{\emph{Kurze Beschreibung wie z.\,B: Tails begrüßte Marianne mit ...}} Sie öffnete mit einem Doppelklick das Terminal-Programm von Tails. \added{Für das was sie vorhatte, wäre die Klickerei mit der Maus viel zu langsam. Von jetzt an würde sie nur noch auf der Kommandozeile im Terminal arbeiten. Keine Bilder, nur Text.
Aber die Kommandozeile hatte es in sich.}
\deleted{Manche nannten es auch Konsole oder Kommando-Zeile.
Das Terminal-Programm war kein normales Programm, es war das Programm, mit dem man auf eine direkt Weise mit dem Computer umgehen konnte.
Man klickte nicht mit der Maus irgendwelche Funktionen an, man tippte nur Befehle ein und bekam vom Computer Antworten als Text auf dem Bildschirm zurück.
Mehr nicht.
Keine Bilder, nur Text.
Aber das Programm hatte es in sich.}
Alle Hacker, die sie kannte, arbeiteten fast ausschließlich \replaced{im Terminal }{mit diesem Programm}\replaced{mit mächtigen Befehlen}{.
Es gab Befehle} für alles, was man machen wollte.
\deleted[remark={Das wird der Leser bald alles lesen.}]{Man konnte damit zum Beispiel in Computer auf der anderen Seite der Erde eindringen, dort Dateien aufstöbern, wiederum auf andere Computer auf anderen Kontinenten kopieren, seine Spuren verwischen, von dort zum nächsten Computer springen.}
\replaced{Im}{Mit dem} Terminal\deleted{-Programm} ging alles viel schneller und genauer als mit den anderen Programmen.

Und das war jetzt wichtig: schnell sein\replaced{, g}{.
G}enau sein.
% chapter 2019_marianne (end)
